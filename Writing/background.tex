\section{Background}
\subsection{Statistical Machine Translation}
Machine Translation (MT) is the use of computer program to automatically translate text or speech from one language to another \cite{MT93}. SMT is a machine translation paradigm that uses statistical models to translate a sequence from the source language ($L_S$) to the target language ($L_T$). First, the text in the $L_S$ is tokenized into a sequence \textit{s} of words via a module called Tokenizer. The sequence \textit{s} is then fetched into the Decoder module, which handles the translation by finding the most relevant target sequence \textit{t} with respect to \textit{s}. Formally, the Decoder searches for the target sequence \textit{t} that has the maximum probability:
$$ P\left(t \mid s \right) = \frac{P\left(s \mid t\right) \, P\left(t\right)}{P\left(s\right)} $$
To do that, it utilizes the two models: 1) the language model and 2) the translation model. The language model learns from monolingual corpus of $L_T$ to derive the posibility of feasible sequences in it ($P\left(t\right)$: how likely sequence \textit{t} occurs in $L_T$). On the other hand, the translation model computes the likelihood $P\left(s \mid t\right)$ of mapping between $s$ and $t$. The mapping is calculated by analysising the bilingual dual corpus to learn the alignment between the words/sequences of two languages.

\subsection{Source Code Migration}
Traditionally, SMT is used widely for translating natural languages \cite{smtbook}, but now it has been adapted to use on programing languages. One of its notable application is for the task Code Migration. In the modern software development, software companies often need to develop a software for multiple platforms which require different programing languages. For example, a same mobile application be written in Objective-C for Ios, in C\# for Windows, and in Java for Android. Thus, there is an increasing demand of migrating source code from one programing language to another \cite{Wu2010}. Recently, the use of SMT for Code Migration has achieved great success: \cite{mppSMT}, \cite{phrasalSMT}. In the scope of this paper, we only consider the SMT-based Code Migration system that translates between Java and C\# due to the popularity of these 2 languages. 


\subsection{Bleu}
The goal in developing Bleu is to find an automatic metric to replace human efforts in evaluating machine translation quality. Manual evaluation is time consuming, expensive and not possible for frequent incremental developing tasks of MT system \cite{Papineni2002}. while an automatic metric can be used handly in many frequent tasks of incremental developing MT system. 
Bleu (\underline{b}i\underline{l}ingual \underline{e}valuation \underline{u}nderstudy) uses the modified form of n-grams precision and length difference penalty to evaluate the quality of text generated by MT compared to referenced one.
