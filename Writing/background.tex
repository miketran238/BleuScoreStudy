\section{Background}
\label{sec:background}

\subsection{Statistical Machine Translation}

Machine Translation aims to translate texts or speech from a language to another.
%is the use of computer program to automatically translate text or
%speech from one language to another.
%
Statistical Machine Translation (SMT) is a machine translation
paradigm that uses statistical models to learn to derive the
translation ``rules'' from a training corpus in order to translate a
sequence from the source language ($L_S$) to the target language
($L_T$). The text in the $L_S$ is tokenized into a sequence \textit{s}
of words. The model searches the most relevant target sequence
\textit{t} with respect to \textit{s}. Formally, the model searches
for the target sequence \textit{t} that has the maximum probability:
$$ P\left(t \mid s \right) = \frac{P\left(s \mid t\right) \, P\left(t\right)}{P\left(s\right)} $$

To do that, it utilizes the two models: 1) the language model and 2)
the translation model. The language model learns from monolingual
corpus of $L_T$ to derive the possibility of feasible sequences in it
($P\left(t\right)$: how likely sequence \textit{t} occurs in
$L_T$). On the other hand, the translation model computes the
likelihood $P\left(s \mid t\right)$ of mapping between $s$ and
$t$. The mapping is calculated by analyzing the bilingual dual
corpus to learn the alignment between the words or sequences of two
languages.

\subsection{SMT-based Code Migration}

Traditionally, SMT is used widely for translating natural
languages~\cite{smtbook}. With the success of SMT, several researchers
have adapted it to use on programming languages to migrate source code
from one language to another (called {\em code migration} or {\em
  language migration}). lpSMT~\cite{fse13-nier} is a model that
directly applies Phrasal, a phrase-based SMT tool, to migrate Java
code to C\texttt{\#}. Source code is treated as a sequence of code
tokens and a Java code fragment is migrated into a fragment in
C\texttt{\#}. Despite that the migrated code is textually similar to
the manually migrated code, the percentage of migrated methods that
are semantically incorrect is high (65.5\%).
%
Karaivanov {\em et al.}~\cite{karaivanov14} also follow phrase-based SMT to
migrate C\texttt{\#} to Java. They use prefix grammar to consider only
the phrases that are potentially the beginning of some syntactic
units.

Nguyen {\em et al.}~\cite{ase15} developed mppSMT by using a
divide-and-conquer approach with syntax-directed translation. mppSMT
constructs from the code the sequence of annotations for code token
types and data types. mppSMT then uses phrase-based SMT three times on
three sequences built from source code: lexemes, syntactic and type
annotations. It integrates the resulting translated code at the
lexical level for all syntactic units into a larger code. The type
annotations help with the translation of data and API types.
The divide-and-conquer spirit is similar to that of Sudoh {\em et
  al.}~\cite{sudoh15} in clause translation for texts.
%
codeSMT~\cite{icsme16} improves over lpSMT by using well-defined
semantics in programming languages to build a context to guide the
translation process in SMT. It integrates five types of features
forming the contexts involving the semantic relations among code
tokens including occurrence association among code tokens, data and
control dependencies among program entities, visibility constraints of
entities, and the consistency in declarations and accesses of
variables, fields and methods.


%One of its notable application is for the task Code Migration. In the
%modern software development, software companies often need to develop
%a software for multiple platforms which require different programing
%languages. For example, a same mobile application be written in
%Objective-C for Ios, in C\# for Windows, and in Java for
%Android. Thus, there is an increasing demand of migrating source code
%from one programing language to another \cite{Wu2010}. Recently, the
%use of SMT for Code Migration has achieved great success:
%\cite{mppSMT}, \cite{phrasalSMT}. In the scope of this paper, we only
%consider the SMT-based Code Migration system that translates between
%Java and C\# due to the popularity of these 2 languages.


%\subsection{Bleu}

%The goal in developing Bleu is to find an automatic metric to replace human efforts in evaluating machine translation quality. Manual evaluation is time consuming, expensive and not possible for frequent incremental developing tasks of MT system \cite{Papineni2002}. while an automatic metric can be used handly in many frequent tasks of incremental developing MT system.
%Bleu (\underline{b}i\underline{l}ingual \underline{e}valuation \underline{u}nderstudy) uses the modified form of n-grams precision and length difference penalty to evaluate the quality of text generated by MT compared to referenced one.


\subsection{BLEU metric}

The key goal of BLEU is to find an automatic metric to replace human
efforts in evaluating machine translation quality. Manual evaluation
is time consuming, expensive and not possible for frequent incremental
developing tasks of a SMT system \cite{Papineni2002}, while an
automatic metric can be used. BLEU
(\underline{B}i\underline{L}ingual \underline{E}valuation
\underline{U}nderstudy)~\cite{Papineni2002} uses the modified form of $n$-grams precision
and length difference penalty to evaluate the quality of text
generated by SMT compared to referenced one.
%
BLEU measures translation quality by the accuracy of translating
$n$-grams to $n$-grams with various values of $n$ (phrases to
phrases):
\[BLEU = BP.{e^{\frac{1}{n}(\log {P_1} + ... + \log {P_n})}}\]
where $BP$ is the {\em brevity penalty value}, which equals to 1 if
the total length (i.e., the number of words) of the resulting sentences
is longer than that of the {\em reference sentences} (i.e., the correct
sentences in the oracle). Otherwise, it equals to the ratio between
those two lengths. $P_i$ is the metrics for the overlapping between
the bag of $i$-grams (repeating items are allowed) appearing in the
resulting sentences and that of $i$-grams appearing in the reference
sentences. Specifically, if $S^{i}_{ref}$ and $S^{i}_{trans}$ are the
bags of $i$-grams appearing in the reference code and in the
translation code respectively, $P_i$ = |$S^{i}_{ref}$ $\cap$
$S^{i}_{trans}$| / |$S^{i}_{trans}$|. The value of \code{BLEU} is
between 0--1. The higher it is, the higher the n-grams precision.

Since $P_i$ represents the accuracy in translating phrases
with $i$ consecutive words, the higher the value of $i$ is used, the
better \code{BLEU} measures translation quality. For example, assume
that a translation \code{Tr} has a high $P_1$ value but a
low~$P_2$. That is, \code{Tr} has high word-to-word accuracy but low
accuracy in translating 2-grams to 2-grams (e.g. the word order might
not be respected in the result). Thus, using both $P_1$ and $P_2$ will
measure \code{Tr} better than using only $P_1$. If
translation~sen\-tences are shorter, \code{BP} is smaller and
\code{BLEU} is smaller. If they are too long and more incorrect words
occur, $P_i$ values are smaller, thus, \code{BLEU} is smaller. $P_i$s
are computed for $i$=1--4.
