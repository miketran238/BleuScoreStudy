\section{{\model}-An effective metric for code migration}
\label{sec:proposal}

%, and can reflect well the semantic
%accuracy of translated results, and inexpensive to compute.
%{\model} is then empirically validated on SMT-based code migration
%models to show its effectiveness.

\subsection{Design}

%\emph{Insight 1 :}
%According to Table~\ref{table:correlation} in section 6, we can
%observe a trend of increasing correlation coefficients with Semantic
%scores when higher levels of representations are used for source code
%from lexical tokens to AST, and to PDGs.
%
%there is a trend of increasing correlation coefficient with
%Semantic score when more complex level of source code's
%representations are used.
%Specifically, GVED has the highest correlation, and it is followed by
%TREED and SED, respectively.

In this work, we introduce {\model}, a novel metric to evaluate
SMT-based code migration tools.  From the empirical results
listed in Section~\ref{sec:alternatives}, we can observe that the
metric measuring translated results in a higher level representation
can achieve higher correlation coefficient with semantic score.
% Tien
However, as explained earlier, we cannot use GRS as a sole metric to
evaluate the migrated code because {\em not all resulting migrated
code is sufficiently correct to build the PDGs or even~ASTs}.
%to compute the feature vectors.

To address this problem, we adapt an idea from machine learning,
called {\em ensemble methods}~\cite{ensemble}. Ensemble methods
use multiple learning models to obtain better
predictive performance than could be obtained from any of the
constituent learning models alone.
% Tien
To apply to our problem, we would use the decision made by the most
reliable metric, which is GRS metric. In the case that the most
reliable metric is not available (due to the fact that a PDG cannot
be built from incorrectly migrated code), we would use the next
reliable one, which is TRS. TRS in turn would face the same issue if
the migrated code is not syntactically correct. In this case, we would
resolve to use STS as it is always computable for any given piece of
code with regard to the reference code. In this work, we design 
{\model} as an ensemble expert-based metric. Generally, {\model} is 
computed in the following formula:

%Therefore, those metrics should have priority order when are used in
%evaluating translation results. The Expert System tries to address
%similar problem by inferring through a knowledge base of \lq if-then
%\rq rules, and using the best expert available for the
%problem. Applying the Expert System's idea onto our problem, the best
%metric available should be used when evaluating translation result.


%\emph{Insight 2 :}
%For SMT-based Code Migration systems, a translated method cannot be guaranteed to be compilable or be built into PDG. It leads to the problems that GVED and TREED are not always applicable even though GVED is better than TREED, and TREED is better than SED. The probability of a translated method can be built into PDG is also lower than  the probability it can be compiled because a segment of code needs to be compiled first before building into PDG.  On the other hand, SED can always be computed regardless of translation result or types of Code Migration systems. To sum up, the 3 metrics have different advantages and disadvantages. Therefore, it is natural to combine the 3 metrics to overcome the disadvantages and boost the advantages. Indeed, there is a theory to backup that combination. In machine learning domain, Ensemble theory is to combine multiple classifiers to form a hopefully better one when each different classifier is more suited for a practical problem.

%From the above insight, we design {\model} as an ensemble %expert-based
%metric. {\model} is a combined metric that takes into consideration
%metrics at the lexical, syntactical, and semantic levels of source
%code while aiming to measure the semantic accuracy of translated code
%in a reliable manner. {\model} utilizes the best metric available and
%guarantees to give a reliable semantic accuracy regardless of
%translation results. If the translated code can be built into PDG, we
%calculate {\model} in term of graph similarity. If the translated code
%cannot be built into PDG, however it is syntactically correct, {\model}
%is calculated as syntax tree similarity. Finally, if the translated code
%is not syntactically correct, we take string similarity. 

$$
RUBY(R, T) = \begin{cases}
				GRS(R, T), 	& \mbox{if } PDGs\mbox{ are applicable} \\
				TRS(R, T), 	& \mbox{if } ASTs\mbox{ are applicable} \\
				STS(R, T), 	& \mbox{otherwise}
			\end{cases}
$$
where $R$ and $T$ are the reference and translated code respectively;
$GRS$, $TRS$, and $STS$ were defined in Section~\ref{sec:alternatives}.
%$r$ is
%the reference code, $t$ is the translated code; $r$ and $t$ are tokenized
%into sequences to use in SED, are parsed into ASTs to use in TREED,
%and are built into PDGs to use in GVED.  \makeatletter
%\def\BState{\State\hskip-\ALG@thistlm} \makeatother
%\begin{algorithm}
%\caption{Calculate {\model}}\label{ruby}
%\begin{algorithmic}[1]
%\If { $\mbox{GVED}\left(r,t\right) $ is applicable }
%\State $\mbox{RUBY}\left(r,t\right) = \mbox{GVED}\left(r,t\right) $
%\ElsIf { $\mbox{TREED}\left(r,t\right) $ is applicable }
%\State $\mbox{RUBY}\left(r,t\right) = \mbox{TREED}\left(r,t\right) $
%\Else
%\State $\mbox{RUBY}\left(r,t\right) = \mbox{SED}\left(r,t\right) $
%\EndIf
%\end{algorithmic}
%\end{algorithm}

\subsection{Proposal result}

To evaluate RUBY, we conducted the same experiment with other four
metrics to show the correlation of RUBY with semantic scores on two
model mppSMT and lpSMT. RUBY's efficiency was tested on a large number
of methods translated by those models.

\begin{figure}[t]
\caption{RUBY vs Semantic (lpSMT)}
\centering
\includegraphics{img/rubyvssem_lpSMT.png}
\label{fig:RubySemlpSMT}
\end{figure}

\begin{figure}[t]
\caption{RUBY vs Semantic (mppSMT)}
\centering
\includegraphics{img/rubyvssem_mppSMT.png}
\label{fig:RubySemMppSMT}
\end{figure}

Figures~\ref{fig:RubySemlpSMT} and~\ref{fig:RubySemMppSMT} show the
scatter plots between RUBY and Semantic scores on two models mppSMT
and lpSMT, respectively. As can be seen from the scatter plot for
lpSMT (Figure~\ref{fig:RubySemlpSMT}), there is a moderately strong,
positive, linear association between the two variables with a few
outliers.  Beside, the scatter plot for mppSMT
(Figure~\ref{fig:RubySemMppSMT}) shows a strong, positive, linear
association between RUBY and Semantic score. There are no outliers in
the data. This is the indication that the result is consistent.

In general, RUBY has high correlation with Semantic scores. As our
experiment result, the correlation coefficient between RUBY and
semantic score for model mppSMT is \textbf{0.862} and for the model
lpMSMP is \textbf{0.836}. In statistics, these values indicate a
strong uphill linear relationship between two quantitative
metrics. That means one metric could be predicted by the other with
high confidence. For example, an increase of 0.5, {\model} score can
be interpreted as an increase of 0.4 in term of Semantic score. Based
on that information, developers can tune the system in an incremental
manner.


%Specifically, a correlation of +0.8 implies that when 	

Although RUBY has a strong correlation coefficient, it is still lower
than GVED scores shown in Section~\ref{sec:alternatives} due to the
fact that only a subset of the dataset is applicable. While GVED is
only computed with the migrated code with sufficient semantic
information, RUBY does not depend on the code, even we can not parse
PDGs and ASTs. This implies that there exist a RUBY score for any
given code. On the other hand, in comparison with TREED, SED
and BLEU, RUBY always outperforms the other three metrics.
	    
  			

\subsection{Result Research}

Correlation coefficients are used in statistics to measure how strong a relationship is between two variable space. Concretely, in this case those variable space are Semantic score and RUBY. 


We aim to figure out specifics of a dataset in which RUBY can work well. Because RUBY was created to estimate the semantic of code translation, the dataset in which RUBY works well also is the dataset containing the high correlation between Semantic score and RUBY. 

\subsubsection{The Use of RANSAC}
The RANSAC (Random Sample Consensus) algorithm~\cite{ Fischler:1981:RSC:358669.358692} is a simple, yet powerful, technique that is commonly applied to the task of estimating the parameters of a model, using data that may be contaminated by outliers. RANSAC estimates a global relation that fits the data, while simultaneously classifying the data into inliers (points consistent with the relation) and outliers (points not consistent with the relation). Due to its ability to tolerate a large fraction of outliers, the algorithm is a popular choice for a variety of robust estimation problems.


RANSAC assumes that the training data consists of inliers that can be explained with the model and outliers that are gross-erroneous samples which do not fit the model at all. So using outliers when training the model would increase our final prediction error, as they contain almost no information about the model. Therefore, in our research scope, we use RANSAC as a mean of data classifier which returns a set of points contributing to the correlation of the method (inliers) and another set of points decreasing the correlation of the method (outliers).

\subsubsection{Experiments}
We conducted independtly 10 experiments with RANCSAC running on the same dataset derived from our result tested on model \textbf{XXX}. The data set contains 375 pairs representing RUBY and Semantic score caculated for the set of traslated methods of \textbf{XXX}. (Table~\ref{table:RANSAC_experiments}) shows results of our 10 experiments which contains the number of inlier points and the correlations of those inliers. The number of inliers is almost around 300 out of 375, while the figure for their correlation is at high rate between 0.93 and 0.95. 
According to the outcome, we chose the dataset result with the \textbf{median} of correlation that is the experiment No5 (correlation = 0.954874293) in the table for analyzing and finding out the specifics of that dataset. Figure~\ref{fig:inliers_outliers} shows the result of experiment No5 gained from RANCSAC.
\begin{table}
	\caption{RANSAC experiments}
	\begin{tabular}{|c|c|c|c|c|}
		\hline
		Experiment Number & Number of inliers & Correlations of inliers \\
		\hline
		10	& 315	& 0.934700727 \\		
		8	& 312	& 0.945424846 \\	
		9	& 312	& 0.947983773 \\
		4	& 310	& 0.948196013 \\
		{\cellcolor[gray]{.8}}5	& {\cellcolor[gray]{.8}}308	& {\cellcolor[gray]{.8}}0.954874293 \\
		7	& 244	& 0.955112418 \\	
		1	& 283	& 0.957661146 \\
		3	& 303	& 0.958122675 \\
		2	& 298	& 0.958659967 \\
		6	& 277	& 0.959811603 \\		
		\hline
	\end{tabular}
	\label{table:RANSAC_experiments}
\end{table}

\begin{figure}[t]
	\caption{An example: Dataset in experiment No5 classified into inlier points(green) and outlier point(yellow)}
	\includegraphics[scale=0.4]{img/inliers_outliers.png}
	\centering
	\label{fig:inliers_outliers}
\end{figure}

\subsubsection{Specification category}
\textbf{Inliers.} We manually classify the method set of the inlier points into serveral categories based on the relation between the method and its reference. Table~\ref{table:inliers_result} shows the result of classification for our set of data.
\textbf{TODO: Explain table}


\textbf{Outliers.} The past Stephen Hawking said: "One of the basic rules of the universe is that nothing is perfect. Perfection simply does not exist". Although RUBY metric performces the high correlation in representation semantic score, there exists cases in which RUBY does not work well. With the same manually checking method as done with inliers, the table~\ref{table:outliers_result} shows RUBY's failure in measuring semantic for those phenomenon.
\textbf{TODO: Explain table}
\begin{table}[]
	\centering
	\caption{Inlier classification}
	\label{table:inliers_result}
	\begin{tabular}{|m{1cm}|m{3cm}|m{4cm}}
		\hline
		Type      & Category         & Description                                                                                                                    
		\\
		\hline
		High RUBY & IDENTIDFIED           & Code is the same with the reference                                                                                             \\
		& SAME\_FUNC\_DIFF\_LEX & Code uses a variety expression for the same functionarity with the reference, e.g for vs foreach, array[i] vs array.get(i)\\
		& SAME\_DATA\_FLOW      & Code has the same data flow with the reference but there are still incorrect pieces of code                                     \\
		\hline
		Low RUBY  & SAME\_KEYWORDS        & Code has the same keywords with the reference, i.e API names, method calls, variables. However, their usage is incorrect           \\
		& DISORDERED            & Code has some same pieces of code with the reference. However they are disordered                                               \\
		& DIFFERENT             & Code is totally different from the reference \\
		\hline                                                                                   
	\end{tabular}
\end{table}


\begin{table}[]
	\centering
	\caption{Outlier classification}
	\label{table:outliers_result}
	\begin{tabular}{|m{1cm}|m{3cm}|m{4cm}}
		\hline
		Type      & Category         & Description                                                                                                                    
		\\
		\hline
		High RUBY &     INCOMPLETED       &  Code has a lot of syntax errors and it is not completed however it still has a majority of same piece with the reference which include unimportant information \\                                   
		\hline
		Low RUBY  & ALGORITHM\_VARIETY        & Code has the same functionarity with the reference, however it has been implemented by other algorithm, the lexical and data flow is almost different           \\
		\hline                                                                                   
	\end{tabular}
\end{table}
\subsection{Threats to Validity}

The dataset of 9 subject projects with +34K method pairs that we used
in our experiments might not be representative. However, it has been
used to evaluate SMT-based code migration models in existing
work~\cite{fse13,ase15,icsme16}. For {\model}, it is impossible to
perform the experiments on all migration models from any programming
language to another. We mitigate this threat by using it on the most
popular SMT-based migration models and from Java to C\texttt{\#}. Our
semantic scores were built by a human subject, which might make
mistakes. Our statistical validation methods might have statistical
errors.




%BLEU has always been doing this and that....

%Researchers usually claim that an improvement in BLEU also meant an
%improvement in translation quality. So

%BLEU has been used for not only evaluating the result but also tuning
%and developing SMT-based migration system.
%%From the results in section 5, BLEU did not reflect the semantic accuracy of source code. --> We need a better metrics to replace BLEU
%However, from the results in section 5, it can be concluded that BLEU
%did not reflect well the semantic accuracy of migrated source code
%since it has weak relation with human judgments. Therefore, we need a
%better metric in order to fit better with programming languages and
%code migration systems.
%%
%%Needed metric should be
%%Reflect semantical meaning of sources code.
%%Automated
%%Low computation's cost
%%Independent of programming language
%%Independent of MT model's type
%Such a metric should have the following requirements:
%
%
%\emph{1}. The metric is more suitable for source code than BLEU. It
%can reflect the semantics of source code and the semantic accuracy of
%translation result. To fulfill that, the metric must have high
%correlation with human evaluation for the translation result.
%
%
%\emph{2}. The metric can be computed automatically and inexpensively in
%order to support the evaluation of migration results in the
%incremental development of SMT-based code migration systems. For
%example, the metric can be calculated quickly after each iteration of
%development so a system can be evaluated and tuned in a timely~manner.
%
%\emph{3}. The metric is independent of the programming languages and
%of the SMT models. A good and reliable metric must have consistent
%results for any languages and models so it can be applied universally
%to any SMT-based code migration systems.
%
%%What is Ruby and why Ruby is good.
%Considering all above requirements, we introduce {\model}, a novel automated metric that can reflect semantic accuracy of translated code. {\model} is also independent of programming languages and machine translation models used in migration system. {\model} measures the semantic accuracy of the resulting code with respect to reference code by comparing their Program Dependence Graph (PDG). PDG captures both the data and control dependencies among program entities. Because those dependencies play an important role in a program, we expect PDG can represent well the semantics of source code. Usually, comparing graph is expensive, but our approach makes it affordable. We estimate the graph edit distance by vectorizing the graph and calculating the vectors edit distance. Because of that, we can save the computational cost and make our approach scalable. Basically, every programming language code can be built into PDG. Hence, our metric can be used for Code Migration systems that migrate different programming languages. Lastly, the way {\model} is measured makes it independent of machine translation models. That means Code Migration systems that deployed different SMT models can still use our metric. Next, we go into details how to formalize the calculation of {\model}
%
%%To reduce the high computational cost, we vectorize the PDGs and calculate the vector difference to estimate the graph difference. This way, we would make sure that our model is practical and applicable in large scaled systems.
%
%When applying MT on source code, there always exists the problem that the translated code is broken in term of syntax. Thus, it is impossible to build PDG or even compile those code. To cope with the problem, our model is designed as best-effort, layered metric  : If the translated code can be built into PDG, we calculate {\model} in term of graph edit distance. If the translated code cannot be built into PDG but is compilable, we calculate {\model} in term of syntax tree edit distance. If the translated code is not compilable, we calculate {\model} in term of string edit distance. We then represent about the 3 metrics graph/tree/string edit distances as follows:



