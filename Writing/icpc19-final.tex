%\documentclass[conference]{IEEEtran}
\documentclass[10pt,conference]{IEEEtran}
\IEEEoverridecommandlockouts
% The preceding line is only needed to identify funding in the first footnote. If that is unneeded, please comment it out.

\usepackage{booktabs}   %% For formal tables:
                        %% http://ctan.org/pkg/booktabs
\usepackage{subcaption} %% For complex figures with subfigures/subcaptions
                        %% http://ctan.org/pkg/subcaption

\usepackage{mathptmx}
\usepackage{microtype}

%\usepackage{nccmath}
\usepackage{amsmath}
\usepackage{newtxmath}

\def\BibTeX{{\rm B\kern-.05em{\sc i\kern-.025em b}\kern-.08em
    T\kern-.1667em\lower.7ex\hbox{E}\kern-.125emX}}

\DeclareSymbolFont{largesymbolsCM}{OMX}{cmex}{m}{n}
\let\txsum\sum
\let\sum\relax
\DeclareMathSymbol{\sum}{\mathop}{largesymbolsCM}{"50}

\DeclareSymbolFont{tienlargesymbolsCM}{OMX}{cmex}{m}{n}
\let\txprod\prod
\let\prod\relax
\DeclareMathSymbol{\prod}{\mathop}{tienlargesymbolsCM}{"51}


\usepackage{balance}

\usepackage{ulem}

\usepackage[table]{xcolor}

\usepackage{listings}
\normalem
%\usepackage{latex8}
%\usepackage{times}
\usepackage{epsf}
\usepackage{ctable}
%\usepackage{latexsym}
%\usepackage{tweaklist}
%\usepackage{rotating}
%\usepackage{listings}
%\usepackage{alltt}
%\usepackage{fvrb-ex}
\usepackage{graphicx}
\usepackage{url}
\usepackage{float}
\usepackage{multicol}
%\floatstyle{boxed}
\restylefloat{figure}
%\usepackage{hyperref}
\usepackage{comment}
\usepackage{paralist}
\usepackage{cite}


\usepackage{xspace}
\newcommand{\cf}{\hbox{\emph{cf.}}\xspace}
\newcommand{\deletia}{\ldots [deletia] \ldots}
\newcommand{\etal}{\hbox{\emph{et al.}}\xspace}
\newcommand{\eg}{\hbox{\emph{e.g.,}}\xspace}
\newcommand{\ie}{\hbox{\emph{i.e.,}}\xspace}
\newcommand{\st}{\hbox{\emph{s.t.}}\xspace}
\newcommand{\wrt}{\hbox{\emph{w.r.t.}}\xspace}
\newcommand{\viz}{\hbox{\emph{viz.}}\xspace}

\newcommand{\model}{\textsc{RUBY}\xspace}

\newcommand{\todo}[1]{\textcolor{red}{TODO: #1}\PackageWarning{TODO:}{#1!}}

%\usepackage{amsmath}
%\usepackage{booktabs}

\usepackage{multirow}
\usepackage{textcomp}

\usepackage[T1]{fontenc}
\usepackage{array}

\usepackage{fancyhdr}
\usepackage[yyyymmdd,hhmmss]{datetime}
\usepackage{algorithm}
\usepackage[noend]{algpseudocode}

\newtheorem{Definition}{Definition}
\newtheorem{Claim}{Claim}
\newtheorem{Lemma}{Lemma}
\newtheorem{Theorem}{Theorem}
\newtheorem{Property}{Property}

\newcommand{\code}[1]{{\small\textsf{#1}}}

%\newcommand{\hoan}[1]{{\color{green!70!black}\textbf{Hoan:}~#1}\xspace}
%\newcommand{\danny}[1]{{\color{blue!70!white}\textbf{Danny:}~#1}\xspace}
%\newcommand{\tien}[1]{{\color{violet!70!white}\textbf{Tien:}~#1}\xspace}
%\newcommand{\michael}[1]{{\color{cyan!70!white}\textbf{Michael:}~#1}\xspace}



\newcommand{\NumChanges}{\textcolor{black}{322K}\xspace}
\newcommand{\NumFiles}{\textcolor{black}{1M}\xspace}
\newcommand{\NumChangeFiles}{\textcolor{black}{291K}\xspace}
\newcommand{\NumChangesIntelliJ}{\textcolor{black}{XXX}\xspace}
\newcommand{\NumFilesIntelliJ}{\textcolor{black}{XXX}\xspace}
\newcommand{\NumDevelopers}{\textcolor{black}{108}\xspace}

\newcommand{\NumDevelopersSecondCorpus}{\textcolor{black}{13K}\xspace}
\newcommand{\NumberSLOCs}{\textcolor{black}{164M}\xspace}
\newcommand{\NumberChangeSLOCs}{\textcolor{black}{81M}\xspace}
\newcommand{\NumbersofChangeGraphNodes}{\textcolor{black}{3M}\xspace}


\newcommand{\NumRepos}{\textcolor{black}{88}\xspace}
\newcommand{\NumPatterns}{\textcolor{black}{17K}\xspace}
\newcommand{\NumRequests}{\textcolor{black}{451}\xspace}
\newcommand{\NumResponses}{\textcolor{black}{108}\xspace}
\newcommand{\RepeatedChangePatterns}{\textcolor{black}{XXX}\xspace}

\newcommand{\NumAllDevelopers}{\textcolor{black}{170,000+}\xspace}
\newcommand{\NumAllProjects}{\textcolor{black}{5,000+}\xspace}


\lstset{
    language={Java}, emph={},
    mathescape=false, escapeinside={/*@}{@*/},
    basicstyle=\scriptsize\sffamily,
    numberstyle=\scriptsize\sffamily,
    emphstyle=\bfseries,
    numbers=left, stepnumber=1, numbersep=-6pt,
    frame=single, xleftmargin=4pt, xrightmargin=4pt, framexleftmargin=0pt, framexrightmargin=0pt,  %xleftmargin=11pt
    columns=flexible, breaklines=true, showspaces=false, showstringspaces=true, showtabs=false, tabsize=2
}

\definecolor{deletedline}{RGB}{255,224,224}
\definecolor{addedline}{RGB}{224,255,224}
\definecolor{modifiedline}{RGB}{231,231,152}

%\lstset{
%    language={}, emph={},
%    mathescape=false, escapechar=@,
%    basicstyle=\scriptsize\sffamily,
%    numberstyle=\scriptsize\sffamily,
%    emphstyle=\bfseries,
%    numbers=left, stepnumber=1, %numbersep=7pt,
%    frame=single, xleftmargin=4pt, xrightmargin=4pt, framexleftmargin=0pt, framexrightmargin=0pt,  %xleftmargin=11pt
%    columns=flexible, breaklines=true, showspaces=false, showstringspaces=true, showtabs=false, tabsize=4
%}

%\def\alignauthor{%
%\end{tabular}%
%  \begin{tabular}[t]{p{0.88\auwidth}}\centering}%

%\makeatletter
%\let\@copyrightspace\relax  % clear copyright block!!!
%\makeatother


\begin{document}

%\title{Paper Title*\\
%{\footnotesize \textsuperscript{*}Note: Sub-titles are not captured in Xplore and
%should not be used}
%\thanks{Identify applicable funding agency here. If none, delete this.}
%}

\title{Does BLEU Score Work for Code Migration?}

\author{\IEEEauthorblockN{Ngoc Tran\IEEEauthorrefmark{1},
Hieu Tran\IEEEauthorrefmark{1},
Son Nguyen\IEEEauthorrefmark{1},
Hoan Nguyen\IEEEauthorrefmark{2}, and
Tien N. Nguyen\IEEEauthorrefmark{1}}
  \IEEEauthorblockA{\IEEEauthorrefmark{1}Computer Science Department, The University of Texas at Dallas, USA,\\ Email: \{nmt140230,trunghieu.tran,sonnguyen,tien.n.nguyen\}@utdallas.edu}
  \IEEEauthorblockA{\IEEEauthorrefmark{2}Computer Science Department,
Iowa State University, USA, Email: hoan@iastate.edu}}

%\author{\IEEEauthorblockN{1\textsuperscript{st} Given Name Surname}
%\IEEEauthorblockA{\textit{dept. name of organization (of Aff.)} \\
%\textit{name of organization (of Aff.)}\\
%City, Country \\
%email address}
%\and
%\IEEEauthorblockN{2\textsuperscript{nd} Given Name Surname}
%\IEEEauthorblockA{\textit{dept. name of organization (of Aff.)} \\
%\textit{name of organization (of Aff.)}\\
%City, Country \\
%email address}
%\and
%\IEEEauthorblockN{3\textsuperscript{rd} Given Name Surname}
%\IEEEauthorblockA{\textit{dept. name of organization (of Aff.)} \\
%\textit{name of organization (of Aff.)}\\
%City, Country \\
%email address}
%\and
%\IEEEauthorblockN{4\textsuperscript{th} Given Name Surname}
%\IEEEauthorblockA{\textit{dept. name of organization (of Aff.)} \\
%\textit{name of organization (of Aff.)}\\
%City, Country \\
%email address}
%\and
%\IEEEauthorblockN{5\textsuperscript{th} Given Name Surname}
%\IEEEauthorblockA{\textit{dept. name of organization (of Aff.)} \\
%\textit{name of organization (of Aff.)}\\
%City, Country \\
%email address}
%\and
%\IEEEauthorblockN{6\textsuperscript{th} Given Name Surname}
%\IEEEauthorblockA{\textit{dept. name of organization (of Aff.)} \\
%\textit{name of organization (of Aff.)}\\
%City, Country \\
%email address}
%}

\maketitle

\begin{abstract}
Machine translation (MT) is a fast growing sub-field of computational linguistic. Until now, the most popular automatic metrics to measure the quality of MT is Bleu score. Lately, MT along with its Bleu metric has been applied to many Software Engineering(SE) tasks. In this paper, we studied Bleu score to validate its suitability for software engineering tasks. We showed that Bleu score does not reflect translation quality due to its weak relation with semantic meaning of the translated source codes. Specifically, an increase in Bleu score does not guarantee an improved in translation quality, and a good translation may have fluctuated Bleu score.  
\end{abstract}


\begin{IEEEkeywords}
Code Migration, Statistical Machine Translation
\end{IEEEkeywords}

\section{Introduction}
\label{sec:intro}

Statistical Machine Translation (SMT)~\cite{smtbook} is a natural
language processing (NLP) approach~that uses statistical learning to
derive the translation ``rules'' from a training data (called a {\em
  corpus}) and applies the trained model to translate a sequence from
the source language ($L_S$) to the target one ($L_T$). SMT produces
translated texts based on the statistical models whose parameters are
trained from a corpus of the corresponding texts in two languages. SMT
has been very successful in translating natural-language texts.
Google Translate~\cite{googletranslate} is a SMT-based tool~that can
accept inputs in 15 natural languages and allows the translation of a
word or a phrase into one of 53 languages.  Microsoft
Translator~\cite{mstranslator} also supports instant translation for
more than 40 languages.

The statistical machine translation community relies on the BLUE
metric (bilingual evaluation understudy) for the purpose of evaluating
SMT models and tools.
%
%what
%
BLUE metric or BLUE sore measures translation quality by the accuracy
of trxanslating $n$-grams to $n$-grams with various values of $n$
(phrases to phrases). BLEU was shown to be highly correlated with
human judgments in natural language MT systems~\cite{Papineni2002}.
Despite the criticism on BLUE metric, it has been remaining one of the
most popular automated and inexpensive metrics to evaluate the quality
of translation models.





Machine Translation (MT) is the use of computer program to translate
text or speech from one language to another. Bleu score evaluates the
quality of MT by calculating the modified n-grams precision and also
taking into account the length difference penalty. Traditionally, MT
is only applied to natural language, but now it is also used for
technical and programming language. One notable use of MT for SE tasks
is Code Migration. Even with that adaptation, SE community still
relies on Blue to evaluate the quality of MT. It is well known that
there is a significant difference between natural language and
programing language: programing language has structure, and
well-defined syntax. This leads to a question as whether Blue score is
suitable for SE task (Code Migration) or not. If it is, we could
continue to use it. Otherwise, we need another metric that is more
suitable for programing language. Hence, the answer to the question
above will help researchers and developers build and evaluate MT-based
Code Migration system better. Some has attempted to answer the
question by stating informal arguments toward the use of Bleu for SE
task \cite{}. However, up to date, there has not been any empirical
evidences to formally address the problem.

Bleu measures the lexical difference between machine generated code and referenced one. On the other hand, to measure the semantic similarity between them is the ultimate goal when evaluating quality of Code Migration system. 
 

Bleu was proved to be correlated with human judgments in natural language MT systems \cite {Papineni02}. However, Callison at el argued that we should not over-rely on Bleu score as an improvement in Bleu score is not sufficient nor necessary to show an in improvement in translation quality \cite {Callison06}. To validate the use of Bleu on SE tasks, we set up an experiment to manually judge the result of multiple MT systems and compare its to the Bleu score. Our result showed that Bleu score has weak correlation to human judgments across 

%\vspace{0.03in} {\em 1.}  \textbf{\code{BLEU}}. This is a popular
%metrics in SMT that measures translation quality by the accuracy of
%translating $n$-grams to $n$-grams with various values of $n$ (phrases
%to phrases):

% \[\code{BLEU} = BP.{e^{\frac{1}{n}(\log {P_1} + ... + \log {P_n})}}~\cite{bleu}\]
%where $BP$ is the {\em brevity penalty value}, which equals to 1 if
%the total length (i.e. the number of words) of the resulting sentences
%is longer than that of the {\em reference sentences} (i.e. the correct
%sentences in the oracle). Otherwise, it equals to the ratio between
%those two lengths. $P_i$ is the metrics for the overlapping between
%the bag of $i$-grams (repeating items are allowed) appearing in the
%resulting sentences and that of $i$-grams appearing in the reference
%sentences. Specifically, if $S^{i}_{ref}$ and $S^{i}_{trans}$ are the
%bags of $i$-grams appearing in the reference code and in the
%translation code respectively, $P_i$ = |$S^{i}_{ref}$ $\cap$
%$S^{i}_{trans}$|/|$S^{i}_{trans}$|. The value of \code{BLEU} is
%between 0-1. The higher it is, the higher the translation quality.

%Since $P_i$ represents the accuracy in translating phrases
%with $i$ consecutive words, the higher the value of $i$ is used, the
%better \code{BLEU} measures translation quality. For example, assume
%that a translation \code{Tr} has a high $P_1$ value but a
%low~$P_2$. That is, \code{Tr} has high word-to-word accuracy but low
%accuracy in translating 2-grams to 2-grams (e.g. the word order might
%not be respected in the result). Thus, using both $P_1$ and $P_2$ will
%measure \code{Tr} better than using only $P_1$. If
%translation~sen\-tences are shorter, \code{BP} is smaller and
%\code{BLEU} is smaller. If they are too long and more incorrect words
%occur, $P_i$ values are smaller, thus, \code{BLEU} is smaller. $P_i$s
%are computed for $i$=1-4.

\section{Background}
\label{sec:background}

\subsection{Statistical Machine Translation}

Machine Translation aims to translate texts or speech from a language to another.
%is the use of computer program to automatically translate text or
%speech from one language to another.
%
Statistical Machine Translation (SMT) is a machine translation
paradigm that uses statistical models to learn to derive the
translation ``rules'' from a training corpus in order to translate a
sequence from the source language ($L_S$) to the target language
($L_T$). The text in the $L_S$ is tokenized into a sequence \textit{s}
of words. The model searches the most relevant target sequence
\textit{t} with respect to \textit{s}. Formally, the model searches
for the target sequence \textit{t} that has the maximum probability:
$$ P\left(t \mid s \right) = \frac{P\left(s \mid t\right) \, P\left(t\right)}{P\left(s\right)} $$

To do that, it utilizes the two models: 1) the language model and 2)
the translation model. The language model learns from the monolingual
corpus of $L_T$ to derive the possibility of feasible sequences in it
($P\left(t\right)$: how likely sequence \textit{t} occurs in
$L_T$). On the other hand, the translation model computes the
likelihood $P\left(s \mid t\right)$ of mapping between $s$ and
$t$. The mapping is calculated by analyzing the bilingual dual
corpus to learn the alignment between the words or sequences of two
languages.

\subsection{SMT-based Code Migration}

Traditionally, SMT is used widely for translating natural
languages~\cite{smtbook}. With the success of SMT, several researchers
have adapted it to use in programming languages to migrate source code
from one language to another (called {\em code migration} or {\em
  language migration}). lpSMT~\cite{fse13-nier} is a model that
directly applies Phrasal, a phrase-based SMT tool, to migrate Java
code to C\texttt{\#}. Source code is treated as a sequence of code
tokens and a Java code fragment is migrated into a fragment in
C\texttt{\#}. Despite that the migrated code is textually similar to
the manually migrated code, the percentage of migrated methods that
are semantically incorrect is high (65.5\%).
%
Karaivanov {\em et al.}~\cite{karaivanov14} also follow phrase-based SMT to
migrate C\texttt{\#} to Java. They use prefix grammar to consider only
the phrases that are potentially the beginning of some syntactic
units.

Nguyen {\em et al.}~\cite{ase15} developed mppSMT by using a
divide-and-conquer approach with syntax-directed translation. mppSMT
constructs from the code the sequence of annotations for code token
types and data types. mppSMT then uses phrase-based SMT three times on
three sequences built from source code: lexemes, syntactic and type
annotations. It integrates the resulting translated code at the
lexical level for all syntactic units into a larger code. The type
annotations help with the translation of data and API types.
The divide-and-conquer spirit is similar to that of Sudoh {\em et
  al.}~\cite{sudoh15} in clause translation for texts.
%
codeSMT~\cite{icsme16} improves over lpSMT by using well-defined
semantics in programming languages to build a context to guide the
translation process in SMT. It integrates five types of features
forming the contexts involving the semantic relations among code
tokens including occurrence association among code tokens, data and
control dependencies among program entities, visibility constraints of
entities, and the consistency in declarations and accesses of
variables, fields, and methods.


%One of its notable application is for the task Code Migration. In the
%modern software development, software companies often need to develop
%a software for multiple platforms which require different programing
%languages. For example, a same mobile application be written in
%Objective-C for Ios, in C\# for Windows, and in Java for
%Android. Thus, there is an increasing demand of migrating source code
%from one programing language to another \cite{Wu2010}. Recently, the
%use of SMT for Code Migration has achieved great success:
%\cite{mppSMT}, \cite{phrasalSMT}. In the scope of this paper, we only
%consider the SMT-based Code Migration system that translates between
%Java and C\# due to the popularity of these 2 languages.


%\subsection{Bleu}

%The goal in developing Bleu is to find an automatic metric to replace human efforts in evaluating machine translation quality. Manual evaluation is time consuming, expensive and not possible for frequent incremental developing tasks of MT system \cite{Papineni2002}. while an automatic metric can be used handly in many frequent tasks of incremental developing MT system.
%Bleu (\underline{b}i\underline{l}ingual \underline{e}valuation \underline{u}nderstudy) uses the modified form of n-grams precision and length difference penalty to evaluate the quality of text generated by MT compared to referenced one.


\subsection{BLEU metric}

The goal of (\underline{B}i\underline{L}ingual \underline{E}valuation
\underline{U}nderstudy)~\cite{Papineni2002} (BLEU) is to find an
automated metric to replace human efforts in evaluating machine
translation quality. Manual evaluation is time consuming, expensive
and not possible for frequent developing tasks of SMT
models~\cite{Papineni2002}, while an automatic metric can be
used. BLEU uses the modified form of $n$-grams precision and length
difference penalty to evaluate the quality of text generated by SMT
compared to referenced one.
%
BLEU measures translation quality by the accuracy of translating
$n$-grams to $n$-grams with various values of $n$ (phrases to
phrases):
\[BLEU = BP.{e^{\frac{1}{n}(\log {P_1} + ... + \log {P_n})}}\]
where $BP$ is the {\em brevity penalty value}, which equals to 1 if
the total length (\ie the number of words) of the resulting sentences
is longer than that of the {\em reference sentences} (\ie the correct
sentences in the oracle). Otherwise, it equals to the ratio between
those two lengths. $P_i$ is the metrics for the overlapping between
the bag of $i$-grams (repeating items are allowed) appearing in the
resulting sentences and that of $i$-grams appearing in the reference
sentences. Specifically, if $S^{i}_{ref}$ and $S^{i}_{trans}$ are the
bags of $i$-grams appearing in the reference code and in the
translation code respectively, $P_i$ = |$S^{i}_{ref}$ $\cap$
$S^{i}_{trans}$| / |$S^{i}_{trans}$|. The value of \code{BLEU} is
between 0--1. The higher it is, the higher the n-grams precision.

Since $P_i$ represents the accuracy in translating phrases
with $i$ consecutive words, the higher the value of $i$ is used, the
better \code{BLEU} measures translation quality. For example, assume
that a translation \code{Tr} has a high $P_1$ value but a
low~$P_2$. That is, \code{Tr} has high word-to-word accuracy but low
accuracy in translating 2-grams to 2-grams (e.g. the word order might
not be respected in the result). Thus, using both $P_1$ and $P_2$ will
measure \code{Tr} better than using only $P_1$. If
translation~sen\-tences are shorter, \code{BP} is smaller and
\code{BLEU} is smaller. If they are too long and more incorrect words
occur, $P_i$ values are smaller, thus, \code{BLEU} is smaller. $P_i$s
are computed for $i$=1--4.

\section{Hypothesis and Research Question}
BLEU was proved to be correlated with human judgments in natural language MT systems \cite {Papineni02}, but no one has confirmed its validity for programming languages. There are two reasons to be concern about the use of BLEU on SMT-based Code Migration system. First, programming language is much difference from natural language: Programming language is written for machines. So it has well-defined semantic and is unambiguous. Programming allows some variations, but the meaning is rigorous. On the other hand, natural languages is ambiguous, and has more relaxed semantics. Moreover, the naturalness of programming language is to give instructions to computer. Hence, it has strict structure and syntactic while natural language is more loosy in that aspect to enable creativity, poetry and metarphors. Second, there is a gap between the purpose of Bleu and the task of evaluating translated source code. BLEU measures the lexical precision of machine generated source code. However, when evaluating translated source code, it is more important to consider the functionality of the generated code. The closer in term of functionality between the translated source code (T) and the reference source code(R), the better quality of translation is. Statically speaking, functionality of sources code is represented semantically by program dependence
graph (PDG) as data and control dependencies among program entities. As a result, BLEU would fail to capture the semantic similarity between resulting code and reference code.

Because of the two intuitive reasons above, we came up with our hypothesis: 
\textbf{ BLEU score does not measure well the similarity in term of semantics between the reference and migrated source code }
In this work, we aim to answer these questions:

\textbf{RQ1: }
Does bleu score reflect the semantic similarity between resulting code and reference code?

\textbf{RQ2: } 
Is Bleu correlated to Lexical (measured by String Edit Distance) representation of code ? If no, can Lexical score represent semantic of code?

\textbf{RQ3: } 
Is Bleu correlated to Syntaxtical (measured by Tree Edit Distance)  representation of code? If no, can Syntax score represent semantic of code?


\section{Methodology}
\subsection{Proof of Hypothesis}
\subsection{Data Collection}
\subsection{Settings and Metrics}

\section{Empirical Results on BLEU Scores}
\label{sec:bleuresult}

%In this section, we present the empirical results to validate our
%hypothesis that \textit{BLEU is not effective in evaluating
%  translation quality of source code migration task}.

\subsection{Correlation between BLEU scores and semantic scores}
To justify the first part of our hypothesis, BLEU score does not reflect well
the semantic accuracy of results translated by a particular model,
we show the relation between BLEU scores and human judgments via semantic scores.
We use Pearson's correlation coefficient~\cite{geek_2015} to gauge
how strong their relation is. The correlation coefficient has a value
between [-1, 1], where 1 indicates the strongest positive relation, -1
indicates the strongest negative relation, and 0 indicates no relation.

%TODO BLEU, n-gram, n=???
Fig.~\ref{fig:BleuSemlpSMT} and Fig.~\ref{fig:BleuSemMppSMT} show the
scatter plots between two metrics: BLEU and Semantic. Each point
represents the scores of a pair of methods where its $x$-axis value is
for BLEU scores and $y$-axis value is for semantic scores. The
correlation coefficient between BLEU and semantic scores for the model
mppSMT is 0.523 and for the model lpSMT is 0.570. These positive values
are closer to 0.5 than to 1.0. This means there is a {\em positive but weak
relation} between BLEU score and semantic score. The weak correlations %help me to check grammar
between the metrics on the results translated by lpSMT and mppSMT are
demonstrated in Fig.~\ref{fig:BleuSemlpSMT} and Fig.~\ref{fig:BleuSemMppSMT}.

%\emph{Observation 1:}

%TODO mark what we want to talk about on figures

\begin{figure}
\caption{BLEU scores vs Semantic scores (lpSMT model)}
\centering
\includegraphics[width=2.9in]{img/bleuvssemantic_lpSMT.png}
\label{fig:BleuSemlpSMT}
\end{figure}

\begin{figure}
\caption{BLEU scores vs Semantic scores (mppSMT model)}
\centering
\includegraphics[width=2.9in]{img/bleuvssemantic_mppSMT.png}
\label{fig:BleuSemMppSMT}
\end{figure}

\subsubsection{{\bf Higher BLEU does not necessarily reflect higher~semantic accuracy}}

In Fig.~\ref{fig:BleuSemlpSMT}, for many specific values of BLEU, the
corresponding semantic scores can spread out for a wide range. For
instance, the BLEU score of 0.75, the corresponding semantic scores
are from 0.25 to 1.
% \textbf{Ngoc: can u mark in the figure please}.
Thus, from this observation, we conclude that the results migrated by
these models with {\em high BLEU scores might not achieve high semantic
scores}.
%There are two reasons for this.
%

%In our sample set, these results can fall in two main cases.
There are two main reasons for this result in our dataset.  First, the
translated methods might have multiple correct phrases, but in an
incorrect order, those methods can be incorrect, even not compilable.
%useless and justified as so in human judgment.
%
For example, in Fig.~\ref{fig:issueexample2}, the translated method
misplaces the position of '\{', making the method have a low
semantic score, however, it has high BLEU score.
%
%Another reason for this implication is that resulting method does not capture the important
%program elements.
In other cases, the migrated results are incomplete code missing the
elements that are trivial for the translation model, yet important
with respect to the syntactic rules of the target language. For
example, the result contains mostly keywords and separators such as
\code{if}, \code{public}, \code{()}, but misses out the important
program elements such as function calls or variable names. In this
case, it will have low semantic score while having a moderate to high
BLEU score. These circumstances indicate the weakness of BLEU metric
in evaluating the translated results in programming language where
syntax rules are well-defined.



%\emph{Observation 2:} For a fixed value of Semantic score, there can
%be many associated BLEU values. Specifically, in the model lpSMT, with
%a Semantic Score of 1, the BLEU scores can vary greatly between 0-1,
%which is reflected on the top horizontal line of dots in the
%Figure~\ref{fig:BleuSemlpSMT}. Similarly, in
%Figure~\ref{fig:BleuSemMppSMT}, with the Semantic Score of 1, the BLEU
%scores are in the range of 0.5 to 1.

\subsubsection{{\bf Higher semantic accuracy is not necessarily reflected by
higher BLEU score}} In the results translated by mppSMT
(Fig.~\ref{fig:BleuSemMppSMT}), for a particular value of the semantic
score, there can be many corresponding BLEU values spreading out in
a~wide range. Specifically, with the semantic score of 1, the BLEU
scores can vary from 0.5 to 1.0. Thus, it can be concluded
that for this model, the migrated code achieving {\em higher semantic score does not necessarily have higher BLEU~score}.

%From this observation, it can be implied that translated code can have
%low BLEU score, but high Semantic score. This can be explained by two
%reasons.
From our data, we observe that there are two main reasons leading to
that result.
%
First, a translated method can use code structure different from the
reference one to perform the same
functionality. Fig.~\ref{fig:mppSMT_example} shows an example that got
a maximum semantic score, but has low BLEU score of 0.4. In this
example, the translated method uses a \code{for} loop instead of a
\code{foreach} loop as in the reference code. The second reason is the
whitespace issue. For example, in Fig.~\ref{fig:mppSMT_example}, the
translated code has the tokens \code{IsSimilar(}, but the reference
code has \code{IsSimilar (}. The former is interpreted as one token
while the latter is interpreted as two. This situation reduces
the precision on phrases, but the human subject still evaluated the
result with a high semantic score. By this experiment, we also
empirically verify that the focus on the lexical precision of BLEU
makes it unable to capture other code-related aspects such as program
dependencies that contribute to program semantics. This might lead
to the ineffectiveness of BLEU in reflecting the semantic accuracy of
translated results.

In brief, {\em BLEU does not reflect well the
semantics of source code, and it is not suitable for evaluating the 
semantic accuracy of the result from an SMT-based code migration model}.

%TODO split translated results
\begin{figure}[t]
\centering
%\begin{lstlisting}[basicstyle=\small\sffamily, stepnumber=1, numbers=left, language=Java, aboveskip=1pt,  belowskip=1pt, numbersep=-5pt]
\lstinputlisting[basicstyle=\scriptsize\sffamily,language=Java]{mppExample.cs}
%\end{lstlisting}
\caption{Translated results and the corresponding reference code}
\label{fig:mppSMT_example}
\end{figure}


\subsection{The Use of BLEU in Comparing Models}
\subsubsection{Theoretical Models}
To prove the second part of our hypothesis, we conduct study to see if an improvement in BLEU score over models reflects the improvement in translation quality represented by human judgments of semantic score. For a same set of 375 original Java methods, the two models mppSMT and mppSMT2 generates two sets of 375 translated methods. Each set has its own BLEU scores and Semantic scores. 
From the 375 pairs, we have 73 pairs that have identical BLEU scores but are different in term of lexical. 26 out of 73 (\textbf{35\%}) have lower semantic score. On average, semantic score of model mppSMT2 is lower than mppSMT's by 0.16. Given that our semantic scores have only 5 pivots range from 0 to 1, a deduction of 0.16 is a large margin as it nearly reduces the semantic score by one pivot point.  
The model mppSMT2 would swap the positions of incorrect phrases in term of lexical. However, those phrases may still have semantic information as the model mppSMT is capable of produce code that are semantically correct but lexically incorrect. There are many cases that they are semantically incorrect but still help the whole translated code syntactically correct or provide users with good starting point for the migration task. We have showed that a theoretical model can achieve identical BLEU score but has lower translation quality compared to another. 
\subsubsection{Practical Models}
Using the same experiment set up, we have two sets of 375 translated methods from two models GNMT and mppSMT. From the sample, we choose pairs of results such that their GNMT's BLEU scores are higher than mppSMT's BLEU scores. There are 215 of such pairs. Among them, 79 pairs have lower semantic score. Then, we perform t-test with alpha confident level of 0.95 on that subset to see if their GNMT's semantic scores are also significantly higher or not. Our null hypothesis is: GNMT's semantic score are higher than mppSMT's one with confident level of 0.95. The results show a t-value of \textbf{-9.1} which is lower than the critical point of -1.97. It meant we would reject the null hypothesis that GNMT's semantic scores are higher than mppSMT's ones. Our result shows that an improvement in BLEU score does not lead to an improvement in translation quality. 
%To validate the use of BLEU in comparing different SMT-based code migration systems, we conduct study to see if an improvement in BLEU score over cross models reflects the improvement in translation quality represented by human judgments of semantic score between those models. For a same set of 375 original Java methods, the two models GNMT and mppSMT generates two sets of 375 translated methods. Each set has its own BLEU scores and Semantic scores.
%We prove that an improvement in BLEU score does not sufficient nor necessary lead to an improvement in semantic score by showing:
%
%1. From the sample, we choose pairs of results such that their GNMT's BLEU scores are higher than mppSMT's BLEU scores. Then, we perform t-test with alpha confident level of 0.95 on that subset to see if their GNMT's semantic scores are also higher or not. The results show a t-value of ... It meant we would reject the null hypothesis that GNMT's semantic scores are higher than mppSMT's ones. So, it can be concluded that an improvement in BLEU score is not sufficient to achieve a higher semantic score. 
%
%2.  From the sample, we choose pairs of results such that their mppSMT's semantic scores are higher than GNMT's semantic scores. Then, we perform t-test with alpha confident level of 0.95 on that subset to see if their mppSMT's BLEU scores are also higher or not. The results show a t-value of ... It meant we would reject the null hypothesis that mppSMT's BLEU scores are higher than GNMT's ones. So, it can be concluded that an improvement in BLEU score is not necessary to achieve a higher semantic score. 

%We ignored pairs of translated results if they have the same BLEU score or Semantic score (136 of the cases). For the remaining results, we found out that in \textbf{34\%} of the cases, the change in BLEU score contradicts the change in Semantic score. It means an improvement in BLEU score leads to a decrease in Semantic score and vice versa. In other words, if one function is translated by two migration models, one-third of the time, the result which has higher BLEU score actually has lower translation quality than the other.

Because of the results above, BLEU is not reliable to use in comparing different SMT-based migration models.





\section{Alternative metrics}
\label{sec:alternatives}
%Since BLEU did not reflect well the semantic similarity of migrated
%code and reference code, there is a need for an alternative metric
%that can fit better with the task. The nature of the task is to
%compare source code in term of program semantics with respect to a
%programming language.
%
%To compare the two language bases, they must be represented in some
%ways. 
As the previous section shows that BLEU is ineffective in evaluating
translated results of migrated source code, in this section we will 
evaluate the effectiveness of various other metrics. The effectiveness of
a metric is expressed in its abilities to measure the similarity between
the reference and the migrated code in term of program semantics respect to 
a programming language.
%SON: How about the abilities to evaluate the translation quality improvement of models
%While a language is normally composed by three important parts:
%vocabulary, grammar, and meaning; programming language can be composed
%by the three corresponding parts: lexeme, syntax, and semantics. 

In general, a programming language can be composed by the three corresponding parts: 
lexeme, syntax, and semantics that are corresponding with three important 
parts in a normal language: vocabulary, grammar, and meaning.
%Therefore, we consider three representations: tokens, abstract syntax trees
%(ASTs), and program dependence graphs (PDGs) that reflect each of the
%three above parts respectively. 
These parts of a programming language are represented respectively in three 
abstractions: tokens (text), abstract syntax tree (AST), and program 
dependence graph (PDG).
%
Recently, there are a number of clone detection studies, token-based 
\cite{thay}, tree-based \cite{thay}, and graph-based \cite{thay} techniques 
that measure the similarity of source code in these three representations.
These works show that the duplication of source code is detected more 
precisely in the higher level representations from text, AST, to PDG.
The cause of this precision might be the more semantics information that
is stored by the higher abstraction level of source code. Based on this 
intuition, we have the following hypothesis: \textit{the metric that 
measures results in the higher abstraction level reflects better the 
similarity of the translated code and the reference one in term of 
program semantics}

%Comparing source code by using those 
%representations would give us three metrics to measure semantics similarity. 
%Each representation would reflect semantics of source code in certain ways.
%For example, a pair of methods that have identical lexemes would have 
%identical program semantics, but if they are different in term of lexeme, 
%it is uncertain to determine their semantic similarity. Therefore, to fully 
%capture how well these representations reflect semantics accuracy, it is 
%necessary to conduct empirical study that measures the correlation between 
%each metric with semantic score graded by human expert.
%choose one representation for each of the above part to
%compare, in order to heuristically estimate the semantic
%similarity. 
%Specifically, token, AST and PDG are representations of lexemes,
%syntax, and semantics respectively. 
%
%Let us explain the metrics to compare the three above representations.
%
%The metrics to compare those representations are SED, TREED and GVED
%which are defined below. Those metrics are then evaluated to show how
%well they reflect Semantic score.
To validate the hypothesis, we conduct our experiments to evaluate the
effectiveness of three metrics: \textbf{string similarity}, 
\textbf{tree similarity}, and \textbf{graph similarity} that measure 
the results in the text, tree and graph representations of code respectively, 
in reflecting the semantic accuracy of translated results.

\input{alternative-metrics}
 
\subsection{Experimental results on alternative metrics}

%To further study the relations between each of the three above metrics
%and the semantic scores, we conducted several experiments in the same
%manner as the previous study described in
%Section~\ref{sec:bleuresult}. We then present our proposed metric
%in Section~7 based on the results of the following experiments.

To evaluate the abilities of STS, TRS and GRS in reflecting the
semantic accuracy of translated code, we conducted three experiments
for these three metrics in the same manner as the experiment described in
Section~\ref{sec:bleuresult}.

The correlation coefficient results between each of the three metrics
with semantic score are described in
Table~\ref{table:correlation}. These results follow a similar trend:
for each model, the correlation coefficients increase 
according the abstraction levels of source code from text, syntax, to semantic
representations in three metrics. For example, for mppSMT, the correlation
coefficient between STS and semantic score is 0.549, whereas that
for TRS is much greater at 0.820. The highest value is the
correlation coefficient between GRS and semantic score at 0.910. A
reason for this phenomenon is that for a pair of source code, if they
are textually identical, they have the same AST representation, and
the exact-match in AST leads to the equivalence between the
corresponding PDGs. Meanwhile, the equivalence in a higher
abstraction level does not necessarily mean the equivalence in a
lower abstraction. For example, Fig.~\ref{fig:mppSMT_example} shows
two methods with equivalent PDGs, but with textually~different. 

%Despite of the increase of the correlation coefficients, GRS and TRS
%are used to measure subsets of methods in our sample set 
%(table \ref{table:metrics}), and the sizes of GRS are greater than 
%TRS's ones, whereas STS is able to apply for all cases. For example, 
%the applicable sets of methods translated by lpSMT are quite limited, 
%75 and 123 (out of 375 methods) for GRS and TRS respectively. The reason
%for this phenomenon is that to construct the higher level representations,
%the translated results need to satisfy certain syntactic and semantic related
%constraints such as resolvable data and control dependencies.
%\begin{table}
%\centering
%\caption{Numbers of applicable methods of metrics}
%\begin{tabular}{|c|c|c|}
%\hline
%  & GRS & TRS\\
%\hline
%GNMT  & 128 & 155  \\
%\hline
%mppSMT  & 239 & 292 \\
%\hline
%lpSMT & 75 & 123 \\
%\hline
%\end{tabular}
%\label{table:metrics}
%\end{table}

Based on these empirical results, we conclude that the metric that
measures the results in the higher abstraction level, the better
metric for reflecting the semantics accuracy. However, the higher
representation like PDG cannot always be constructed due to the
missing syntactic or semantic-related information in the translated
results that is required to construct ASTs and PDGs. For example, the
sets of methods that are translated by lpSMT, and applicable for GRS
and TRS are quite limited, 75 and 123 (out of 375 methods),
respectively (the respective numbers for mppSMT are 239 and 292, and
those for GNMT are 128 and 155). Therefore, in
Section~\ref{sec:proposal}, we propose a metric to evaluate the
quality of translated code, which uses the measurement of the
similarity of translated code and expected results in multiple
representations at different abstraction levels.

%It is not worth to use SED instead of BLEU since it still has the same
%problems as BLEU while its advantage is insignificant.
%We then present our proposed metric in section~7 based on the results of the following experiments.

%To verify the 3 metrics SED, TREED, and GVED , we conducted an exploratory study that reveals their correlation with the ground truth Semantic score. Table \ref{table:correlation} summarizes our results for the two models mppSMT and lpSMT. We will explain each metric \rq s correlation in details as follows:

\begin{table}
\centering
\caption{Correlation between each metric with  semantic score on three models}
\begin{tabular}{|c|c|c|c|c|}
\hline
 & STS & TRS & GRS\\
\hline
mppSMT  & 0.549 & 0.820 & 0.910 \\
\hline
lpSMT  & 0.533 & 0.786 & 0.823 \\
\hline
GNMT & 0.692 & 0.734 & 0.927 \\
\hline
\end{tabular}
\label{table:correlation}
\end{table}


%\subsubsection{\textbf{SED vs Semantic}}
%
%SED is similar to BLEU in the way that both of them compare source
%code in term of lexical representation. Indeed, the two metrics have
%similar correlations with Semantic score
%(Table~\ref{table:correlation}). The correlation coefficient between SED and Semantic score is 0.549,
% 0.533, and 0.692 for three models mppSMT, lpSMT, and GNMT respectively.
%%equivalent 0.523 and 0.549 (mppSMT), and 0.67 and 0.675 (lpSMT). SED
%SED suffers the same drawbacks as BLEU: it does not take into
%consideration the structures of source code, and it compares source
%code only in term of lexical tokens.
%It is not worth to use SED instead of BLEU since it still has the same
%problems as BLEU while its advantage is insignificant.

%\begin{figure}
%\caption{SED vs Semantic (lpSMT)}
%\centering
%\includegraphics{img/sedvssem_lpSMT.png}
%\label{fig:SedSemlpSMT}
%\end{figure}
%
%\begin{figure}
%\caption{SED vs Semantic (mppSMT)}
%\centering
%\includegraphics{img/sedvssem_mppSMT.png}
%\label{fig:SedSemMppSMT}
%\end{figure}

%\subsubsection{\textbf{TREED vs Semantic}}
%TREED compares source code at higher level of representation (Syntax Tree). Syntax of source code is the pre-requisite before mentioning about its semantics or functionality. Comparing methods in term of syntax is likely to reflect semantic accuracy better than comparing at lexical level. This argument is proved by our empirical results:

%Technically, a translated method cannot be said to perform any functionality if it cannot be compiled. However, in the Code Migration problem, a translated method which has wrong syntax can still be useful for developers.

%Figure \ref{fig:TREEDmppSMT} shows the scatter plots between 2
%metrics: TREED and Semantic score for the model mppSMT. In general, the result has similar
%trend as in the relation of BLEU and Semantic score: the data points
%are too scattered to show a strong correlation and there are
%several outliers. For a fixed value of Semantic score, TREED score can
%still vary in a large range. However, compared to BLEU, the variation
%is much smaller. For example, a pair of method that has Semantic score
%of 0.5 can possibly have TREED scores in range of 0.7 to 1 while such
%range is 0.5 to 1 for BLEU. TREED shows noticeable improvement over
%BLEU or SED on the correlation with Semantic score on the mppSMT model
%(0.549 to 0.820), on lpSMT model (0.533 to 0.786), and on the GNMT model (0.692 to 0.734).

%\emph{Observation 1:} For a fixed value of Semantic score, there can be many associated TREED values. Specifically, in the model lpSMT, with a Semantic Score of 1, the TREED scores can be varied greatly between 0-1, which was reflected on the top horizontal line of dots in figure \ref{fig:TREEDlpSMT}. Similarly, in the figure \ref{fig:TREEDmppSMT}, with a Semantic Score of 1, the TREED scores are in the range of 0.7 to 1.
%
%\emph{Observation 2:} For a fixed value of TREED, there can be many associated Semantic scores. For example, the figure \ref{fig:TREEDlpSMT} shows that for a high TREED score, for example 0.8, can have Semantic Score from 0.25 to 1. This can be observed by the vertical line of dots in the figure.



% Comparing figure \ref{fig:BleuSemMppSMT} and figure \ref{fig:TREEDmppSMT}, it can be realized that on the model mppSMT, those horizontal lines of dots in figure \ref{fig:BleuSemMppSMT} became shorter in figure \ref{fig:TREEDmppSMT}. It means the variation of TREED score for certain Semantic score is lower. Data points in the figure can also be approximately fitted with a regression line even though there still are some outliers.

%From observation 1, it can be implied that a translated method can have low TREED score, but high Semantic score. On the other hand, from observation 2, a translated method can have high TREED score, but low Semantic score. The two implications above shows that an improvement in TREED is not sufficient nor necessary to improve translation migration quality. However, from observation 3, there is hint of positive improvement that using TREED would reflect Semantic score better than BLEU.

%Issues of TREED that were showed by the figure \ref{fig:TREEDmppSMT} can be explained by two reasons. First, a translated method can be syntactically correct, however still does not have the same functionality as the reference
%code (high TREED score, low Semantic score). Secondly, there
%exists the scenarios of low TREED score with high Semantic score. For
%example, a translated method can have an incorrect place for a
%semicolon, which makes it not compiled. Beside that mistake, if it can
%reflect the functionality of the reference code, it still has a high
%Semantic score. However, due to the increase of coefficient, there is
%an indication that TREED would reflect syntactic correctness better
%than BLEU.

%In certain circumstance, TREED could be used to evaluate SMT-based
%Migration system that focuses on translating correct syntax code.

%\begin{figure}
%\caption{TREED vs Semantic (lpSMT)}
%\centering
%\includegraphics{img/treed_lpSMT.png}
%\label{fig:TREEDlpSMT}
%\end{figure}
%
%\begin{figure}
%\caption{TREED vs Semantic (mppSMT)}
%\centering
%\includegraphics{img/treed_mppSMT.png}
%\label{fig:TREEDmppSMT}
%\end{figure}
%
%\subsubsection{\textbf{GVED vs Semantic}}

%There are studies that compare PDGs to measure the semantic
%similarity. PDGs capture all data and control dependence of program
%elements, and those dependencies are the keys to reflect functionality
%of source code. Therefore, GVED is expected to have the best
%correlation with Semantic score. Our results cement this argument.

%Figure \ref{fig:GVEDmppSMT} shows the scatter plots between 2 metrics:
%GVED and Semantic score when GVED is applicable. There are 240 of such
%points for the model mppSMT in the total of 375 pairs of methods, and
%75 points for lpSMT, respectively. The correlation coefficients between
%GVED and Semantic score are 0.910, 0.823, and 0.927 for the 3 models mppSMT, lpSMT, and GNMT respectively. These values show the better correlations with
%Semantic score than any of the BLEU, SED, and TREED metrics.
%
%significant improvements on both models comparing to any of the other
%3 metrics.
%
%All other three metrics have correlation coefficients with Semantic
%Score less than 0.7 while GVED achieves remarkable correlation
%coefficients of nearly 1.0 .
%
%GVED's promising result makes it an obvious choice to evaluate
%SMT-based Code Migration systems.
%

%A caveat to this is that not all migrated code is sufficiently correct
%to build the corresponding PDGs.
%
%the translated methods have too many errors that cannot be built into
%PDG, or even be compiled.
%That explains the situation in which the number of data points
%available for model lpSMT is too small to draw conclusion about the
%correlation with high confidence. Therefore, even though GVED is a
%metric with highest correlation with Semantic scores, it is not always
%applicable. To cope with its limitation while still utilizing its
%strength in correlation with semantic scores, we explore the
%combination of GVED and other metrics in our novel metric, {\model}.


%\begin{figure}
%\caption{GVED vs Semantic (lpSMT)}
%\centering
%\includegraphics{img/gved_lpSMT.png}
%\label{fig:GVEDlpSMT}
%\end{figure}
%
%\begin{figure}
%\caption{GVED vs Semantic (mppSMT)}
%\centering
%\includegraphics{img/gved_mppSMT.png}
%\label{fig:GVEDmppSMT}
%\end{figure}


\section{Proposed Metric}
%\subsection{\model}

%BLEU has always been doing this and that....

%Researchers usually claim that an improvement in BLEU also meant an
%improvement in translation quality. So 

%BLEU has been used for not only evaluating the result but also tuning
%and developing SMT-based migration system.
%%From the results in section 5, BLEU did not reflect the semantic accuracy of source code. --> We need a better metrics to replace BLEU
%However, from the results in section 5, it can be concluded that BLEU
%did not reflect well the semantic accuracy of migrated source code
%since it has weak relation with human judgments. Therefore, we need a
%better metric in order to fit better with programming languages and
%code migration systems.
%%
%%Needed metric should be 
%%Reflect semantical meaning of sources code.
%%Automated
%%Low computation's cost
%%Independent of programming language
%%Independent of MT model's type
%Such a metric should have the following requirements:
%
%
%\emph{1}. The metric is more suitable for source code than BLEU. It
%can reflect the semantics of source code and the semantic accuracy of
%translation result. To fulfill that, the metric must have high
%correlation with human evaluation for the translation result.
%
%
%\emph{2}. The metric can be computed automatically and inexpensively in
%order to support the evaluation of migration results in the
%incremental development of SMT-based code migration systems. For
%example, the metric can be calculated quickly after each iteration of
%development so a system can be evaluated and tuned in a timely~manner.
%
%\emph{3}. The metric is independent of the programming languages and
%of the SMT models. A good and reliable metric must have consistent
%results for any languages and models so it can be applied universally
%to any SMT-based code migration systems.
%
%%What is Ruby and why Ruby is good.
%Considering all above requirements, we introduce {\model}, a novel automated metric that can reflect semantic accuracy of translated code. {\model} is also independent of programming languages and machine translation models used in migration system. {\model} measures the semantic accuracy of the resulting code with respect to reference code by comparing their Program Dependence Graph (PDG). PDG captures both the data and control dependencies among program entities. Because those dependencies play an important role in a program, we expect PDG can represent well the semantics of source code. Usually, comparing graph is expensive, but our approach makes it affordable. We estimate the graph edit distance by vectorizing the graph and calculating the vectors edit distance. Because of that, we can save the computational cost and make our approach scalable. Basically, every programming language code can be built into PDG. Hence, our metric can be used for Code Migration systems that migrate different programming languages. Lastly, the way {\model} is measured makes it independent of machine translation models. That means Code Migration systems that deployed different SMT models can still use our metric. Next, we go into details how to formalize the calculation of {\model}
%
%%To reduce the high computational cost, we vectorize the PDGs and calculate the vector difference to estimate the graph difference. This way, we would make sure that our model is practical and applicable in large scaled systems. 
%
%When applying MT on source code, there always exists the problem that the translated code is broken in term of syntax. Thus, it is impossible to build PDG or even compile those code. To cope with the problem, our model is designed as best-effort, layered metric  : If the translated code can be built into PDG, we calculate {\model} in term of graph edit distance. If the translated code cannot be built into PDG but is compilable, we calculate {\model} in term of syntax tree edit distance. If the translated code is not compilable, we calculate {\model} in term of string edit distance. We then represent about the 3 metrics graph/tree/string edit distances as follows:


 {\model} can be calculated as follows: 
%if (GVED(s,t) != -1) RUBY(s,t) = GVED(s,t)\\
%else if (TREED(s,t) != -1) RUBY(s,t) = TREED(s,t)\\
%else RUBY(s,t) = SED(s,t)

%$\mbox{RUBY}\left(s,t\right) = \begin{cases} 
%\mbox{GVED}\left(s,t\right), & \mbox{if } \mbox{GVED}\left(s,t\right) \mbox{ is applicable}\\ 
%\mbox{TED}\left(s,t\right), & \mbox{if } n\mbox{ is odd} 
%\end{cases}$

\makeatletter
\def\BState{\State\hskip-\ALG@thistlm}
\makeatother
\begin{algorithm}
\caption{Calculate {\model}}\label{euclid}
\begin{algorithmic}[1]
\If { $\mbox{GVED}\left(r,t\right) $ is applicable }
\State $\mbox{RUBY}\left(r,t\right) = \mbox{GVED}\left(r,t\right) $
\ElsIf { $\mbox{TREED}\left(r,t\right) $ is applicable }
\State $\mbox{RUBY}\left(r,t\right) = \mbox{TREED}\left(r,t\right) $
\Else 
\State $\mbox{RUBY}\left(r,t\right) = \mbox{SED}\left(r,t\right) $
\EndIf
%\Procedure{MyProcedure}{}
%\State $\textit{stringlen} \gets \text{length of }\textit{string}$
%\State $i \gets \textit{patlen}$
%\BState \emph{top}:
%\If {$i > \textit{stringlen}$} \Return false
%\EndIf
%\State $j \gets \textit{patlen}$
%\BState \emph{loop}:
%\If {$\textit{string}(i) = \textit{path}(j)$}
%\State $j \gets j-1$.
%\State $i \gets i-1$.
%\State \textbf{goto} \emph{loop}.
%\State \textbf{close};
%\EndIf
%\State $i \gets i+\max(\textit{delta}_1(\textit{string}(i)),\textit{delta}_2(j))$.
%\State \textbf{goto} \emph{top}.
%\EndProcedure
\end{algorithmic}
\end{algorithm}

\section{Related Work}

There exist
many studies aiming to measure the functionality similarity of source
code, which utilize the similarities of structures and
dependencies~\cite{clone-tse07,roy09,baker97,ccfinder,cpminer,deckard,deckard2,horwitz01}.
%baxter98,ducasse99
However, they are not reliable as their results sometimes contradict
with human judgments on semantic accuracy~\cite{deckard2}.

However, there exists criticism on BLEU
as Callison-Burch {\em et al.}~\cite{Callison} argued that an
improvement in BLEU metric is not sufficient nor necessary to show an
improvement in translation quality. Despite such criticism,

\section{Conclusion}
In conclusion, in this work, we invalidated the use of BLEU for code migration. We showed counterexamples to prove that BLEU is not effective in reflecting the semantic accuracy of translated code and in comparing different SMT-based migration systems. Our empirical study illustrated that BLEU has weak correlation with human judgment in the task of measuring semantics of translated source code. Also, BLEU's conclusion about the difference between models' translation quality is inconsistent with semantic score's one. We concluded that BLEU is not suitable for source code migration, and proposed a new alternative metric {\model} to replace BLEU. {\model} is a novel ensemble metric that takes into account three representation levels of source code. We validated {\model} on multiple practical Code Migration systems to show its reliability in estimating semantic accuracy and comparing translation quality between systems. 

\section*{Acknowledgment}
This work was supported in part by the US National Science
Foundation (NSF) grants CCF-1723215, CCF-1723432, TWC-1723198,
CCF-1518897, and CNS-1513263.

%\section*{Acknowledgment}

%The preferred spelling of the word ``acknowledgment'' in America is without 
%an ``e'' after the ``g''. Avoid the stilted expression ``one of us (R. B. 
%G.) thanks $\ldots$''. Instead, try ``R. B. G. thanks$\ldots$''. Put sponsor 
%acknowledgments in the unnumbered footnote on the first page.

\newpage

\balance
%\bibliographystyle{abbrv}
%\citestyle{acmauthoryear}

\bibliographystyle{IEEEtran}

%\setcitestyle{numbers,sort&compress}
\bibliography{ase18}


%\section*{References}

%Please number citations consecutively within brackets \cite{b1}. The 
%sentence punctuation follows the bracket \cite{b2}. Refer simply to the reference 
%number, as in \cite{b3}---do not use ``Ref. \cite{b3}'' or ``reference \cite{b3}'' except at 
%the beginning of a sentence: ``Reference \cite{b3} was the first $\ldots$''

%Number footnotes separately in superscripts. Place the actual footnote at 
%the bottom of the column in which it was cited. Do not put footnotes in the 
%abstract or reference list. Use letters for table footnotes.

%Unless there are six authors or more give all authors' names; do not use 
%``et al.''. Papers that have not been published, even if they have been 
%submitted for publication, should be cited as ``unpublished'' \cite{b4}. Papers 
%that have been accepted for publication should be cited as ``in press'' \cite{b5}. 
%Capitalize only the first word in a paper title, except for proper nouns and 
%element symbols.

%For papers published in translation journals, please give the English 
%citation first, followed by the original foreign-language citation \cite{b6}.

%\begin{thebibliography}{00}
%\bibitem{b1} G. Eason, B. Noble, and I. N. Sneddon, ``On certain integrals of Lipschitz-Hankel type involving products of Bessel functions,'' Phil. Trans. Roy. Soc. London, vol. A247, pp. 529--551, April 1955.
%\bibitem{b2} J. Clerk Maxwell, A Treatise on Electricity and Magnetism, 3rd ed., vol. 2. Oxford: Clarendon, 1892, pp.68--73.
%\bibitem{b3} I. S. Jacobs and C. P. Bean, ``Fine particles, thin films and exchange anisotropy,'' in Magnetism, vol. III, G. T. Rado and H. Suhl, Eds. New York: Academic, 1963, pp. 271--350.
%\bibitem{b4} K. Elissa, ``Title of paper if known,'' unpublished.
%\bibitem{b5} R. Nicole, ``Title of paper with only first word capitalized,'' J. Name Stand. Abbrev., in press.
%\bibitem{b6} Y. Yorozu, M. Hirano, K. Oka, and Y. Tagawa, ``Electron spectroscopy studies on magneto-optical media and plastic substrate interface,'' IEEE Transl. J. Magn. Japan, vol. 2, pp. 740--741, August 1987 [Digests 9th Annual Conf. Magnetics Japan, p. 301, 1982].
%\bibitem{b7} M. Young, The Technical Writer's Handbook. Mill Valley, CA: University Science, 1989.
%\end{thebibliography}
%\vspace{12pt}
%\color{red}
%IEEE conference templates contain guidance text for composing and formatting conference papers. Please ensure that all template text is removed from your conference paper prior to submission to the conference. Failure to remove the template text from your paper may result in your paper not being published.


\end{document}
