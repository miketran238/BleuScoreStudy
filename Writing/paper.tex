        \documentclass[sigconf,review, anonymous]{acmart}
        \acmConference[ESEC/FSE 2018]{The 26th ACM Joint European Software Engineering Conference and Symposium on the Foundations of Software Engineering}{4-9 November, 2018}{Lake Buena Vista, Florida, United States}

\usepackage{booktabs} % For formal tables


\begin{document}
\title{Bleu Score Study}



\author{Ngoc M. Tran}
\orcid{1234-5678-9012}
\affiliation{%
  \institution{University of  Texas at Dallas}
  \streetaddress{800 E. Campbell Rd}
  \city{Richardson}
  \state{Texas}
  \postcode{75080}
}
\email{ngoctran@utdallas.edu}




\begin{abstract}
Machine translation (MT) is a fast growing sub-field of computational linguistic. Until now, the most popular automatic metrics to measure the quality of MT is Bleu score. Lately, MT along with its Bleu metric has been applied to many Software Engineering(SE) tasks. In this paper, we studied Bleu score to validate its suitability for software engineering tasks. We showed that Bleu score does not reflect translation quality due to its weak relation with semantic meaning of the translated source codes. Specifically, an increase in Bleu score does not guarantee an improved in translation quality, and a good translation may have fluctuated Bleu score.  
\footnote{More abstract}
\end{abstract}

%
% The code below should be generated by the tool at
% http://dl.acm.org/ccs.cfm
% Please copy and paste the code instead of the example below.
%


\keywords{ACM proceedings, \LaTeX, text tagging}


\maketitle

\section{Research Questions and Hypothesis}
\subsection{RQ1}
Does bleu score reflect semantic meaning of translated source code?
\subsection{RQ2} 
If the answer to RQ1 is 'no', is Bleu correlated to Lexical representation of code?
\subsection{RQ3} 
If the answer to RQ1 is 'no', is Bleu correlated to Syntaxtical representation of code?
\subsection{Our hypothesis}
Our hypothesis is that bleu score does not measure well the closeness in term of semantics between the reference and translated source code. 
\section{Methodology}
\subsection{Proof of Hypothesis}
\subsection{Data Collection}
\subsection{Settings and Metrics}

\section{Evaluation}
\emph{2.} \textbf{String Edit Distance (SED):} This metric measures
effort that a user must edit in term of the code tokens
that need to be deleted/added in order to transform the
resulting code into the correct one. It is computed as:  $SED = \frac{EditDistance\left(s_R, s_T\right)}{length\left(s_R\right)}$ where $EditDistance\left(s_R, s_T\right)$ is the editing distance between each pair of the reference method $s_R$ and the translated method $s_T$; and the denominator is the total length of the referenced method. The metrics is also normalized in 0-1 range.

\emph{3.} \textbf{Tree Edit Distance (TREED):} This metric measures the difference between the Abstract Syntax Trees (AST) of referenced method and translated method. Specifically, the tree edit distance between two trees is calculated by number of operations (add/delete/replace/move) to make them identical. \cite{algorithm}. 
It is computed as:  $TREED = \frac{TreeEditDistance\left(AST_R, AST_T\right)}{CountNodes \left(AST_R\right)}$ where $TreeEditDistance\left(AST_R, AST_T\right)$ is the editing distance between two trees of referenced method $AST_R$ and the translated method $AST_T$; and the denominator is the total nodes in the tree of the referenced method.  The metrics is also normalized in 0-1 range.

\section{Proposal}
\section{Related Works}

\section{Conclusions}
This paragraph will end the body of this sample document.
Remember that you might still have Acknowledgments or
Appendices; brief samples of these
follow.  There is still the Bibliography to deal with; and
we will make a disclaimer about that here: with the exception
of the reference to the \LaTeX\ book, the citations in
this paper are to articles which have nothing to
do with the present subject and are used as
examples only.
%\end{document}  % This is where a 'short' article might terminate





\bibliographystyle{ACM-Reference-Format}
\bibliography{reference}

\end{document}
