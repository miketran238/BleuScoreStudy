% This is "sig-alternate.tex" V2.1 April 2013 
% This file should be compiled with V2.5 of "sig-alternate.cls" May 2012
%
% This example file demonstrates the use of the 'sig-alternate.cls'
% V2.5 LaTeX2e document class file. It is for those submitting
% articles to ACM Conference Proceedings WHO DO NOT WISH TO
% STRICTLY ADHERE TO THE SIGS (PUBS-BOARD-ENDORSED) STYLE.
% The 'sig-alternate.cls' file will produce a similar-looking,
% albeit, 'tighter' paper resulting in, invariably, fewer pages.
%
% ----------------------------------------------------------------------------------------------------------------
% This .tex file (and associated .cls V2.5) produces:
%       1) The Permission Statement
%       2) The Conference (lo cation) Info information
%       3) The Copyright Line with ACM data
%       4) NO page numbers
%
% as against the acm_proc_article-sp.cls file which
% DOES NOT produce 1) thru' 3) above.
%
% Using 'sig-alternate.cls' you have control, however, from within
% the source .tex file, over both the CopyrightYear
% (defaulted to 200X) and the ACM Copyright Data
% (defaulted to X-XXXXX-XX-X/XX/XX).
% e.g.
% \CopyrightYear{2007} will cause 2007 to appear in the copyright line.
% \crdata{0-12345-67-8/90/12} will cause 0-12345-67-8/90/12 to appear in the copyright line.
%
% ---------------------------------------------------------------------------------------------------------------
% This .tex source is an example which *does* use
% the .bib file (from which the .bbl file % is produced).
% REMEMBER HOWEVER: After having produced the .bbl file,
% and prior to final submission, you *NEED* to 'insert'
% your .bbl file into your source .tex file so as to provide
 % ONE 'self-contained' source file.
%
% ================= IF YOU HAVE QUESTIONS =======================
% Questions regarding the SIGS styles, SIGS policies and
% procedures, Conferences etc. should be sent to
% Adrienne Griscti (griscti@acm.org)
%
% Technical questions _only_ to
% Gerald Murray (murray@hq.acm.org)
% ===============================================================
%
% For tracking purposes - this is V2.0 - May 2012

\PassOptionsToPackage{table}{xcolor}
\documentclass[sigconf,review,anonymous]{acmart}

%% Some recommended packages.
\usepackage{booktabs}   %% For formal tables:
                        %% http://ctan.org/pkg/booktabs
\usepackage{subcaption} %% For complex figures with subfigures/subcaptions
                        %% http://ctan.org/pkg/subcaption

\usepackage{mathptmx}
\usepackage{microtype}

%\usepackage{nccmath}
\usepackage{amsmath}
\usepackage{newtxmath}

\DeclareSymbolFont{largesymbolsCM}{OMX}{cmex}{m}{n}
\let\txsum\sum
\let\sum\relax
\DeclareMathSymbol{\sum}{\mathop}{largesymbolsCM}{"50}

\DeclareSymbolFont{tienlargesymbolsCM}{OMX}{cmex}{m}{n}
\let\txprod\prod
\let\prod\relax
\DeclareMathSymbol{\prod}{\mathop}{tienlargesymbolsCM}{"51}


\usepackage{balance}

\usepackage{ulem}

\usepackage[table]{xcolor}

\usepackage{listings}
\normalem
%\usepackage{latex8}
%\usepackage{times}
\usepackage{epsf}
\usepackage{ctable}
%\usepackage{latexsym}
%\usepackage{tweaklist}
%\usepackage{rotating}
%\usepackage{listings}
%\usepackage{alltt}
%\usepackage{fvrb-ex}
\usepackage{graphicx}
\usepackage{url}
\usepackage{float}
\usepackage{multicol}
%\floatstyle{boxed}
\restylefloat{figure}
\usepackage{hyperref}
\usepackage{comment}
\usepackage{paralist}

\usepackage{xspace}
\newcommand{\cf}{\hbox{\emph{cf.}}\xspace}
\newcommand{\deletia}{\ldots [deletia] \ldots}
\newcommand{\etal}{\hbox{\emph{et al.}}\xspace}
\newcommand{\eg}{\hbox{\emph{e.g.,}}\xspace}
\newcommand{\ie}{\hbox{\emph{i.e.,}}\xspace}
\newcommand{\st}{\hbox{\emph{s.t.}}\xspace}
\newcommand{\wrt}{\hbox{\emph{w.r.t.}}\xspace}
\newcommand{\viz}{\hbox{\emph{viz.}}\xspace}

\newcommand{\todo}[1]{\textcolor{red}{TODO: #1}\PackageWarning{TODO:}{#1!}}

%\usepackage{amsmath}
%\usepackage{booktabs}

\usepackage{multirow}
\usepackage{textcomp}

\usepackage[T1]{fontenc}
\usepackage{array}

\usepackage{fancyhdr}
\usepackage[yyyymmdd,hhmmss]{datetime}

\newtheorem{Definition}{Definition}
\newtheorem{Claim}{Claim}
\newtheorem{Lemma}{Lemma}
\newtheorem{Theorem}{Theorem}
\newtheorem{Property}{Property}

\newcommand{\code}[1]{{\small\textsf{#1}}}

%\newcommand{\hoan}[1]{{\color{green!70!black}\textbf{Hoan:}~#1}\xspace}
%\newcommand{\danny}[1]{{\color{blue!70!white}\textbf{Danny:}~#1}\xspace}
%\newcommand{\tien}[1]{{\color{violet!70!white}\textbf{Tien:}~#1}\xspace}
%\newcommand{\michael}[1]{{\color{cyan!70!white}\textbf{Michael:}~#1}\xspace}



\newcommand{\NumChanges}{\textcolor{black}{322K}\xspace}
\newcommand{\NumFiles}{\textcolor{black}{1M}\xspace}
\newcommand{\NumChangeFiles}{\textcolor{black}{291K}\xspace}
\newcommand{\NumChangesIntelliJ}{\textcolor{black}{XXX}\xspace}
\newcommand{\NumFilesIntelliJ}{\textcolor{black}{XXX}\xspace}
\newcommand{\NumDevelopers}{\textcolor{black}{108}\xspace}

\newcommand{\NumDevelopersSecondCorpus}{\textcolor{black}{13K}\xspace}
\newcommand{\NumberSLOCs}{\textcolor{black}{164M}\xspace}
\newcommand{\NumberChangeSLOCs}{\textcolor{black}{81M}\xspace}
\newcommand{\NumbersofChangeGraphNodes}{\textcolor{black}{3M}\xspace}


\newcommand{\NumRepos}{\textcolor{black}{88}\xspace}
\newcommand{\NumPatterns}{\textcolor{black}{17K}\xspace}
\newcommand{\NumRequests}{\textcolor{black}{451}\xspace}
\newcommand{\NumResponses}{\textcolor{black}{108}\xspace}
\newcommand{\RepeatedChangePatterns}{\textcolor{black}{XXX}\xspace}

\newcommand{\NumAllDevelopers}{\textcolor{black}{170,000+}\xspace}
\newcommand{\NumAllProjects}{\textcolor{black}{5,000+}\xspace}


\lstset{
    language={Java}, emph={},
    mathescape=false, escapeinside={/*@}{@*/},
    basicstyle=\small\sffamily,
    numberstyle=\small\sffamily,
    emphstyle=\bfseries,
    numbers=left, stepnumber=1, numbersep=-6pt,
    frame=single, xleftmargin=4pt, xrightmargin=4pt, framexleftmargin=0pt, framexrightmargin=0pt,  %xleftmargin=11pt
    columns=flexible, breaklines=true, showspaces=false, showstringspaces=true, showtabs=false, tabsize=2
}

\definecolor{deletedline}{RGB}{255,224,224}
\definecolor{addedline}{RGB}{224,255,224}
\definecolor{modifiedline}{RGB}{231,231,152}

%\lstset{
%    language={}, emph={},
%    mathescape=false, escapechar=@,
%    basicstyle=\scriptsize\sffamily,
%    numberstyle=\scriptsize\sffamily,
%    emphstyle=\bfseries,
%    numbers=left, stepnumber=1, %numbersep=7pt,
%    frame=single, xleftmargin=4pt, xrightmargin=4pt, framexleftmargin=0pt, framexrightmargin=0pt,  %xleftmargin=11pt
%    columns=flexible, breaklines=true, showspaces=false, showstringspaces=true, showtabs=false, tabsize=4
%}

%\def\alignauthor{%
%\end{tabular}%
%  \begin{tabular}[t]{p{0.88\auwidth}}\centering}%

%\makeatletter
%\let\@copyrightspace\relax  % clear copyright block!!!
%\makeatother

\begin{document}

%\special{papersize=8.5in,11in}
\setlength{\pdfpageheight}{\paperheight}
\setlength{\pdfpagewidth}{\paperwidth}

% Copyright
%\setcopyright{acmcopyright}
%\setcopyright{acmlicensed}
%\setcopyright{rightsretained}
%\setcopyright{usgov}
%\setcopyright{usgovmixed}
%\setcopyright{cagov}
%\setcopyright{cagovmixed}


% DOI
%\acmDOI{10.475/123_4}

% ISBN
%\acmISBN{123-4567-24-567/08/06}

\acmConference[FSE 2018]{International Symposium on Foundations of Software Engineering}{November 2018}{Orlando, USA}

%\acmPrice{\$15.00}

%
% --- Author Metadata here ---
%\conferenceinfo{WOODSTOCK}{'97 El Paso, Texas USA}
%\CopyrightYear{2007} % Allows default copyright year (20XX) to be over-ridden - IF NEED BE.
%\crdata{0-12345-67-8/90/01}  % Allows default copyright data (0-89791-88-6/97/05) to be over-ridden - IF NEED BE.
% --- End of Author Metadata ---

%------- TITLE ----------------------------------
%\title{Capture Relevancy Between Texts and API Elements via Software
%  Documentation with Vector Representation}

%Tien
%\title[]{Capturing Relevance Between English Words and API Elements via Vector Representation
%and Applications}

\title[]{Does BLUE Score Work for Source Code?}
%\subtitle {Draft version {\today} \currenttime}

%({\today} \currenttime)

%\numberofauthors{4} %  in this sample file, there are a *total*

% of EIGHT authors. SIX appear on the 'first-page' (for formatting
% reasons) and the remaining two appear in the \additionalauthors section.
%

%------------------------------------------
%\author{Hoan Anh Nguyen}
%\affiliation{Iowa State University, USA}
%\email{hoan@iastate.edu}

%\author{Tien N. Nguyen}
%\affiliation{University of Texas at Dallas, USA}
%\email{tien.n.nguyen@utdallas.edu}

%\author{Danny Dig}
%\affiliation{Oregon State University, USA}
%\email{digd@eecs.oregonstate.edu}

%\author{Michael Hilton}
%\affiliation{Carnegie Mellon University, USA}
%\email{mhilton@cmu.edu}

%---------------------------------------

% There's nothing stopping you putting the seventh, eighth, etc.
% author on the opening page (as the 'third row') but we ask,
% for aesthetic reasons that you place these 'additional authors'
% in the \additional authors block, viz.


%\date{30 July 1999}
% Just remember to make sure that the TOTAL number of authors
% is the number that will appear on the first page PLUS the
% number that will appear in the \additionalauthors section.


\begin{abstract}
Statistical machine translation (SMT) is a fast-growing sub-field of
computational linguistics. Until now, the most popular automatic metric
to measure the quality of SMT is BiLingual Evaluation Understudy
(BLEU) score. Lately, SMT along with the BLEU metric has been applied
to a Software Engineering task named {\em code migration}. 
%
(In)Validating the use of BLEU score could advance the research and
development of SMT-based code migration tools. Unfortunately, there is
no study to approve or disapprove the use of BLEU score for source
code.
%
In this paper, we conducted an empirical study on BLEU score to
(in)validate its suitability for the code migration task due to its
inability to reflect the semantics of source code. In our work, we use
human judgment as the ground truth to measure the semantic correctness of
the migrated code. Our empirical study demonstrates that BLEU
does not reflect translation quality due to its weak correlation with
the semantic correctness of translated code. We provided
counter-examples to show that BLEU is ineffective in comparing the
translation quality between SMT-based models. Due to BLEU's
ineffectiveness for code migration task, we propose an alternative
metric {\model}, which considers lexical, syntactical, and semantic
representations of source code. We verified that {\model} achieves a
higher correlation coefficient with the semantic correctness of migrated
code, 0.775 in comparison with 0.583 of BLEU score. We also confirmed
the effectiveness of {\model} in reflecting the changes in translation
quality of SMT-based translation models. With its advantages, {\model}
can be used to evaluate SMT-based code migration~models.

%The effectiveness of
%{\model} is also confirmed by its consistency with the decisions by the semantic 
%correctness in comparing translation
%models. 
%Statistical machine translation (SMT) is a fast-growing sub-field of
%computational linguistic. Until now, the most popular automatic metric
%to measure the quality of SMT is BiLingual Evaluation Understudy
%(BLEU) score. Lately, SMT along with its BLEU metric has been applied
%to a Software Engineering task named code migration. Unfortunately,
%there is no study to approve or disapprove the use of BLEU score for
%source code. In this paper, we conducted an empirical study on BLEU
%score to (in)validate its suitability for the code migration task
%because of its inability to reflect semantics of source code. In our
%work, we use human judgment as the ground truth to measure the
%semantic correctness of the migrated code. (In)Validating the use of
%BLEU score could advance the research and development of SMT-based
%code migration tools. We provided counter-examples to show that an
%improvement in BLEU is not sufficient nor necessary for an improvement
%in translation quality. Our empirical study also demonstrated BLEU
%score does not reflect translation quality due to its weak correlation
%with the semantic correctness of translated code. Due to BLEU's
%ineffectiveness for code migration task, we propose an alternative
%metric {\model}, which considers lexical, syntactical, and semantic
%representations of source code. We then verified that {\model}
%achieves a high correlation coefficient with the semantic correctness
%of migrated code. With its advantages, {\model} could be used to
%evaluate SMT-based code migration~models.


%Statistical Machine translation (SMT) is a fast growing sub-field of
%computational linguistic. Until now, the most popular automatic metric
%to measure the quality of SMT is BiLingual Evaluation Understudy
%(BLEU) score. Lately, SMT along with its BLEU metric has been applied
%to the Software Engineering(SE) task named Code
%Migration. Unfortunately, there are no study to approve or disapprove
%the use of BLEU for source code. In this paper, we conducted an
%empirical study on BLEU score to (in)validate its suitability for the
%code migration task because of its inability to reflect semantics of
%source code. In our work, we use human judgment as the ground truth to
%measure the semantic correctness of the migrated code. (In)Validating
%the use of BLEU score could advance the research and development of
%SMT-based code migration tools. We provided counter-examples to show
%that an improvement in BLEU is not sufficient nor necessary for an
%improvement in translation quality. Our empirical study also
%demonstrated BLEU score does not reflect translation quality due to
%its weak relation with the semantic correctness of translated code.
%
%Due to BLEU's ineffectiveness for code migration task, we propose an
%alternative metric {\model}, which considers lexical,
%syntactical, and semantic representations of source code. We then
%verified that {\model} could achieve a high correlation coefficient
%with the semantic correctness of migrated code. With its advantages,
%{\model} could be used to evaluate SMT-based code migration tools.
\end{abstract}


%type references, method calls, and field references.

%\renewcommand{\shortauthors}{}

\setcopyright{none}

\settopmatter{printacmref=false, printfolios=false}

%\if@ACM@manuscript{false}

%\ACM@manuscriptfalse

%\printtopmatter{none}

\maketitle



%An important class of software engineering problems is centered around relation between the English text descriptions and
%relevant source code including API elements. A text is relevant to
%source code when it can be used to describe the functionality,
%behaviors, purposes, and related aspects of the code. Existing
%approaches capture the relevance relation between an English word and
%a code element by automatically tokenizing the name of the code/API
%element into English words, and a source file is considered as a
%collection of those words and the words from its comments, and then
%compared with textual software documents. This solution is not always
%ideal since in a program, a developer could have API elements' names
%with different textual values than the words in the text
%descriptions. In this paper, inspired by the Wor2Vec model for NLP,
%we develop a specialized neural network model, {\tool} that
%characterizes the APIs via a low-dimensional continuous vector space
%that encodes information on many contexts in which the APIs
%appear. With vector representations, {\tool} captures the relevance
%relations between English words and API elements.
%%and those amongwords and APIs.
%Our empirical evaluation showed that our vector
%representations reflect human knowledge of the
%relevance between words and API elements. We also demonstrated the usefulness of
%our model in two SE applications: searching API example code and mining single API
%mappings across software libraries.

%API documentation is a very crucial resource for developers in
%understanding various aspects on the API usages in software
%libraries. An interesting nature of such documentation is the presence
%of code elements embedded in natural language texts that explain their
%purposes, usages and mutual connections with others. Some recent work
%has explored that nature for different purposes such as discovering
%code elements in documents and generating summaries for classes and
%methods with context. However, none of the existing approaches is
%capable of further capturing semantic relations of code elements
%(i.e. embedded APIs hereafter) with related words and with other
%relevant APIs at the same time. In this paper, by considering an
%embedded API element as a word, we characterize this API via a
%low-dimensional continuous vector that encodes information on many
%contexts in which that API appears as possible. Our empirical study
%shows that the vector representation learned from a large corpus of
%documentation is capable of capturing semantic relations between API
%elements and words. That is, APIs with similar functionalities have
%similar embeddings; and APIs and related words are close to each other
%in the vector space without explicit word matching. Our experiment
%also suggests that the proposed representation for embedded API
%elements has promising potential for API code search.


\section{Introduction}
\label{sec:intro}

Statistical Machine Translation (SMT)~\cite{smtbook} is a
natural~language processing (NLP) approach that uses statistical
learning to derive the translation ``rules'' from a training data and
applies the trained model to translate a sequence from the source
language ($L_S$) to the target one ($L_T$). SMT produces translated
texts based on the statistical models whose parameters are trained
from a corpus of corresponding texts in two languages. SMT has
been successful in translating natural-language texts.  Google
Translate~\cite{googletranslate} is a SMT-based tool~that can accept
inputs in 15 natural languages and allows the translation of texts
into one of 53 languages. Microsoft Translator~\cite{mstranslator}
also supports instant translation for more than 40~languages.

The statistical machine translation community relies on the BLEU ({\em
  BiLingual Evaluation Understudy}) metric for the purpose of
evaluating SMT models and tools. {\em BLEU metric}, also called {\em
  BLEU~score}, measures translation quality by the accuracy of
translating text phrases to another with various phrases'
lengths. BLEU was shown to be highly correlated with human judgments
on the translated texts from natural-language SMT
tools~\cite{Papineni2002}. 
%
%However, there exists criticism on BLEU as Callison-Burch {\em et
%  al.}~\cite{Callison} argued that an improvement in BLEU metric is
%not sufficient nor necessary to show an improvement in translation
%quality. Despite such criticism, 
BLEU score remains one of the most popular automated and inexpensive
metrics to evaluate the quality of translation models.


%However, Callison at el argued that we should not over-rely on Bleu
%score as an improvement in Bleu score is not sufficient nor necessary
%to show an in improvement in translation quality \cite {Callison06}.

In recent years, several researchers in Software Engineering (SE) and
Programming Languages have been exploring the NLP techniques and
models to build automated SE tools. SMT has been directly used or
adapted to be used to translate/migrate source code in different
programming
languages~\cite{fse13-nier,icse14-demo,karaivanov14,ase15,icsme16}. The
problem is called {\em language migration}. In the modern world of
computing, language migration is important. Software vendors often
develop a software product for multiple operating platforms in
different languages. For example, the same mobile app could be
developed for iOS (in Objective-C), Android (in Java), and Windows
Phone (in C\texttt{\#}).
%Rather than developing each version of a software product
%independently, it would be more economical to develop the product in
%one platform/language and then migrate to another.
Thus, there is an increasing need for migration/translation
of source code from one programming language to another.
%

Unlike natural-language texts, source code follows syntactic rules and
has well-defined semantics with respect to programming languages. A
natural question is {\em how effective BLEU score is in~evaluating the
  results of migrated source code in language migration}. The answer
to this question is important because if it does, we could~establish
an automated metric to evaluate the quality of SMT-based code
migration tools, and otherwise, a relevant question should be raised:
{\em what is an alternative metric?} Unfortulately, there has not yet
any empirical evidence to either validate or invalidate the
effectiveness of BLEU score in applying to source code in language
migration.

Because the BLEU metric measures the the phrase-to-phrase translation
while source code has well syntactic and semantics, we hypothesize
that {\em BLEU metric is not effective in evaluating the translated
  results of migrated source code}. With respect to the use of BLEU
for the translation results by a single model or its use to compare
the results across models, this key hypothesis can further be divided
into two parts: (1) BLEU does not reflect well the semantic accuracy
of migrated source code with regard to the original code when they are
translated by a model, and (2) the improvement of a model over another
cannot be measured by the BLEU metric.
%
We conducted experiments to provide counterexamples to empirically
validate that hypothesis. We choose the language migration task
because it is a popular SE task that applies SMT. 

For the first part, we chose the migration models that focus on phrase
translation, for example, lpSMT~\cite{fse13-nier} that adapts a
phrase-to-phrase translation model~\cite{phrasal10}. This type of
models produces migrated code with high lexical accuracy, i.e., high
correctness for sequences of code tokens. However, several tokens or
sequences of tokens are placed in the incorrect locations.  This
results in an increment in BLEU but with a lower semantic acccuracy in
migrated code. We also chose the migration models/tools that focus on
structures. Specifically, we picked mppSMT tool~\cite{ase15} that has
high semantic accuracy but with a wide range of BLEU
scores. Importantly, we aimed to show that BLEU has weak correlation
with human judgments on translated source code in term of
functionality. For second part, we chose the three most popular
models: lpSMT~\cite{fse13-nier}, mppSMT~\cite{ase15}, and
GNMT~\cite{tien}. We showed that an improvement in BLEU is not
sufficient nor necessary to reflect an improvement in the quality of
code migration.

%  reflect well the semantic accuracy on the migrated source code}
%syntactic and semantics, we hypothesize that {\em BLEU metric does not
%Because the BLEU metric 
%measures the phrase-to-phrase translation while source code has well
%syntactic and semantics, we hypothesize that {\em BLEU metric does not
%  reflect well the semantic accuracy on the migrated source code}. We
%conducted experiments to provide counterexamples to empirically
%validate that hypothesis. We choose the language migration task 
%%from Java to C\texttt{\#} 
%because it is a popular SE task that applies SMT. We show that under
%some circumstances an improvement in BLEU is not sufficient to reflect
%an improvement in code migration quality, and in other circumstances
%that it is not necessary to improve BLEU in order to achieve an
%improvement in migration quality.
%
%Specifically, for the sufficiency, we chose the migration models that
%focus on phrase translation, for example, fpSMT~\cite{fse13-nier} that
%adapts a phrase-to-phrase translation model~\cite{phrasal10}. This
%type of models produces migrated code with high lexical accuracy,
%i.e., high correctness for sequences of code tokens. However, several
%tokens or sequences of tokens are placed in the incorrect locations.
%This results in an improvement in BLEU but with a lower semantic
%acccuracy in migrated code. For the necessary part, we chose the
%migration models/tools that focus on structures. Specifically, we
%picked mppSMT tool~\cite{ase15} that has high semantic accuracy but
%with a wide range of BLEU~scores.

In our experiment, we used a dataset of 34,209 pairs of methods
in 9 projects that were manually migrated from Java to
C\texttt{\#} by developers. The dataset was used for evaluating the
code migration models/tools in existing research~\cite{ase15}. We used
those above SMT-based migration tools to perform code migration for
those methods. We then manually investigated {\bf 375}
randomly-selected pairs of methods. For each pair of the manually
migrated method and the automatically migrated one from a tool, we
computed the BLEU score and assigned a semantic accuracy score. The
semantic accuracy score was given by a developer after examining the
original code in Java, the manually migrated code in C\texttt{\#} and
the auto-migrated code in C\texttt{\#} by a tool. For the first part
of our hypothesis, we then computed the correlation between the BLEU
scores and semantic scores of all the pairs. Our result shows that the
BLEU metric has a weak correlation to the semantic accuracy of the
migrated code. For the second part of our hypothesis, we compared the
translated results of the same set of 375 methods for two different
SMT-based migration models. In nearly half of those methods, an
improvement of BLEU score does not indicate an improvement in semantic
score, and vice-versa.
%Tien
%We propose a metric: GVED that measure the similarity between graph
%representation. The intuition.

In this work, we also introduce, {\model}, a novel metric to evaluate
the results of the SMT-based code migration tools. {\model} focuses on
measuring the accuracy of the semantics of the code with respect to
the reference code in the ground truth. That is, it measures how close
the resulting code to the ground truth when the semantics of the code
is considered. {\model} measures semantics via program dependence
graph (PDG) as data and control dependencies among program entities
are considered as the key elements in a program. We also integrate
three different scores from lexical, syntactical, and semantic levels
into a final {\model} score. The lexical and syntactical scores are
measured via string edit distance and tree edit distance,
respectively, between the resulting code and the reference one. The
intention of the lexical and syntactic scores is for the resulting
code that might {\em not} be lexically or syntactically correct. We
also conducted experiments to evaluate {\model} as in the previous
experiments for BLEU. Our result shows that the new metric {\model} is
highly correlated to the human judgments on the semantic accuracy of
the resulting migrated code. {\model} can also be used to compare
different SMT-based code migration models as it can successfully
measure 95\% of the cases of the change in translation quality. The
contributions of this paper include:

1. An empirical evidence to show that BLEU metric does not reflect
well the semantic accuracy of the migrated code for SMT-based
migration tools.

2. {\model}, a novel metric to evaluate the results of the SMT-based
code migration tools, that integrates the scores at the lexical,
syntactical, and semantic levels in source code.

Our dataset is publicly available for evaluation.


%In this paper we give a number of counterexamples for Bleu��s
%correlation with human judgments. We show that under some
%circumstances an improvement in Bleu is not sufficient to reflect a
%genuine improvement in translation quality, and in other circumstances
%that it is not necessary to improve Bleu in order to achieve a
%noticeable improvement in translation quality.



%Machine Translation (MT) is the use of computer program to translate
%text or speech from one language to another. Bleu score evaluates the
%quality of MT by calculating the modified n-grams precision and also
%taking into account the length difference penalty. Traditionally, MT
%is only applied to natural language, but now it is also used for
%technical and programming language. One notable use of MT for SE tasks
%is Code Migration. Even with that adaptation, SE community still
%relies on Blue to evaluate the quality of MT. It is well known that
%there is a significant difference between natural language and
%programing language: programing language has structure, and
%well-defined syntax. This leads to a question as whether Blue score is
%suitable for SE task (Code Migration) or not. If it is, we could
%continue to use it. Otherwise, we need another metric that is more
%suitable for programing language. Hence, the answer to the question
%above will help researchers and developers build and evaluate MT-based
%Code Migration system better. Some has attempted to answer the
%question by stating informal arguments toward the use of Bleu for SE
%task \cite{}. However, up to date, there has not been any empirical
%evidences to formally address the problem.

%Bleu measures the lexical difference between machine generated code and referenced one. On the other hand, to measure the semantic similarity between them is the ultimate goal when evaluating quality of Code Migration system. 
 

%Bleu was proved to be correlated with human judgments in natural language MT systems \cite {Papineni02}. However, Callison at el argued that we should not over-rely on Bleu score as an improvement in Bleu score is not sufficient nor necessary to show an in improvement in translation quality \cite {Callison06}. To validate the use of Bleu on SE tasks, we set up an experiment to manually judge the result of multiple MT systems and compare its to the Bleu score. Our result showed that Bleu score has weak correlation to human judgments across 
%-----------------------


%\vspace{0.03in} {\em 1.}  \textbf{\code{BLEU}}. BLEU measures
%translation quality by the accuracy of translating $n$-grams to
%$n$-grams with various values of $n$ (phrases to phrases):

% \[\code{BLEU} = BP.{e^{\frac{1}{n}(\log {P_1} + ... + \log {P_n})}}~\cite{bleu}\]
%where $BP$ is the {\em brevity penalty value}, which equals to 1 if
%the total length (i.e. the number of words) of the resulting sentences
%is longer than that of the {\em reference sentences} (i.e. the correct
%sentences in the oracle). Otherwise, it equals to the ratio between
%those two lengths. $P_i$ is the metrics for the overlapping between
%the bag of $i$-grams (repeating items are allowed) appearing in the
%resulting sentences and that of $i$-grams appearing in the reference
%sentences. Specifically, if $S^{i}_{ref}$ and $S^{i}_{trans}$ are the
%bags of $i$-grams appearing in the reference code and in the
%translation code respectively, $P_i$ = |$S^{i}_{ref}$ $\cap$
%$S^{i}_{trans}$|/|$S^{i}_{trans}$|. The value of \code{BLEU} is
%between 0-1. The higher it is, the higher the translation quality.

%Since $P_i$ represents the accuracy in translating phrases
%with $i$ consecutive words, the higher the value of $i$ is used, the
%better \code{BLEU} measures translation quality. For example, assume
%that a translation \code{Tr} has a high $P_1$ value but a
%low~$P_2$. That is, \code{Tr} has high word-to-word accuracy but low
%accuracy in translating 2-grams to 2-grams (e.g. the word order might
%not be respected in the result). Thus, using both $P_1$ and $P_2$ will
%measure \code{Tr} better than using only $P_1$. If
%translation~sen\-tences are shorter, \code{BP} is smaller and
%\code{BLEU} is smaller. If they are too long and more incorrect words
%occur, $P_i$ values are smaller, thus, \code{BLEU} is smaller. $P_i$s
%are computed for $i$=1-4.
 
\section{Introduction}
<<<<<<< HEAD
Machine Translation (MT) is the use of computer program to translate text or speech from one language to another. The most popular automatic metrics to evaluate quality of MT is Bleu score. Traditionally, MT is only applied to natural language, but now it is also used for technical and programming language. One notable usage of MT for SE tasks is code migration. Even with that adapation, SE community still relies on Blue score. This leads to a question as whether Blue score is suitable for SE tasks or not.
Machine Translation (MT) is the use of computer program to translate text or speech from one language to another. Traditionally, MT is only applied to natural language, but now it is also used for technical and programming language. Even with that evolution, SE community still relies on Blue score as the most popular automated metrics to evaluate the quality of MT. This leads to a question as whether Blue score is suitable for SE tasks. 
Bleu score evaluate the quality of MT by calculating the modified n-grams precision and also taking into account the length difference penalty. Bleu was proved to be correlated with human judgments in natural language MT systems \cite {Papineni02}. However, Callison at el argued that we should not over-rely on Bleu score as an improvement in Bleu score is not sufficient nor necessary to show an in improvement in translation quality \cite {Callison06}. To validate the use of Bleu on SE tasks, we set up an experiment to manually judge the result of multiple MT systems and compare its to the Bleu score. Our result showed that Bleu score has weak correlation to human judgments across 

\section{Background}
\subsection{Machine Translation and Code Migration}
\subsection{Metrics}
Bleu (bilingual evaluation understudy) uses the modified form of n-grams precision and length difference penalty to evaluate the quality of text generated by MT compared to referenced one.


\section{Research Questions and Hypothesis}
\subsection{RQ1}
Does bleu score reflect semantic meaning of translated source code?
\subsection{RQ2} 
If the answer to RQ1 is 'no', is Bleu correlated to Lexical representation of code?
\subsection{RQ3} 
If the answer to RQ1 is 'no', is Bleu correlated to Syntaxtical representation of code?
\subsection{Our hypothesis}
Our hypothesis is that bleu score does not measure well the closeness in term of semantics between the reference and translated source code. 
\section{Methodology}
\subsection{Proof of Hypothesis}
\subsection{Data Collection}
\subsection{Settings and Metrics}
\section{Evaluation}
Since Bleu score is not suitable for SE task (code migration), we propose a new metric RUBY to evaluate quality of machine translation. 
\section{Proposal}
\section{Related Works}

\section{Conclusions}
This paragraph will end the body of this sample document.
Remember that you might still have Acknowledgments or
Appendices; brief samples of these
follow.  There is still the Bibliography to deal with; and
we will make a disclaimer about that here: with the exception
of the reference to the \LaTeX\ book, the citations in
this paper are to articles which have nothing to
do with the present subject and are used as
examples only.
%\end{document}  % This is where a 'short' article might terminate






% (36.8\% relatively higher). 
%We also built an Eclipse plugin that automatically adds the import
%statements of the required libraries for a given code snippet and sets
%up the Maven links to the repositories for those libraries.

%\section*{Acknowledgments}
%We would like to thank 
%Sarah Nadi,
%Stas Negara,
%Mihai Codoban,
%Sruti Srinivasa, % Sruti did not give feedback as she was busy with her own CHI submissions
%Zhen Yu,
%Ameya Ketkar,
%and Samantha Khairunnesa 
%for providing valuable feedback on earlier drafts of this paper.

% TODO: add here the Grant numbers

\newpage

\balance
%\bibliographystyle{abbrv}
%\citestyle{acmauthoryear}

\bibliographystyle{ACM-Reference-Format}

\setcitestyle{numbers,sort&compress}

%\setcitestyle{numbers,sort&compress}
\bibliography{fse18}



% That's all folks!
\end{document}
