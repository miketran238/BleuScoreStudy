% This is "sig-alternate.tex" V2.1 April 2013 
% This file should be compiled with V2.5 of "sig-alternate.cls" May 2012
%
% This example file demonstrates the use of the 'sig-alternate.cls'
% V2.5 LaTeX2e document class file. It is for those submitting
% articles to ACM Conference Proceedings WHO DO NOT WISH TO
% STRICTLY ADHERE TO THE SIGS (PUBS-BOARD-ENDORSED) STYLE.
% The 'sig-alternate.cls' file will produce a similar-looking,
% albeit, 'tighter' paper resulting in, invariably, fewer pages.
%
% ----------------------------------------------------------------------------------------------------------------
% This .tex file (and associated .cls V2.5) produces:
%       1) The Permission Statement
%       2) The Conference (lo cation) Info information
%       3) The Copyright Line with ACM data
%       4) NO page numbers
%
% as against the acm_proc_article-sp.cls file which
% DOES NOT produce 1) thru' 3) above.
%
% Using 'sig-alternate.cls' you have control, however, from within
% the source .tex file, over both the CopyrightYear
% (defaulted to 200X) and the ACM Copyright Data
% (defaulted to X-XXXXX-XX-X/XX/XX).
% e.g.
% \CopyrightYear{2007} will cause 2007 to appear in the copyright line.
% \crdata{0-12345-67-8/90/12} will cause 0-12345-67-8/90/12 to appear in the copyright line.
%
% ---------------------------------------------------------------------------------------------------------------
% This .tex source is an example which *does* use
% the .bib file (from which the .bbl file % is produced).
% REMEMBER HOWEVER: After having produced the .bbl file,
% and prior to final submission, you *NEED* to 'insert'
% your .bbl file into your source .tex file so as to provide
 % ONE 'self-contained' source file.
%
% ================= IF YOU HAVE QUESTIONS =======================
% Questions regarding the SIGS styles, SIGS policies and
% procedures, Conferences etc. should be sent to
% Adrienne Griscti (griscti@acm.org)
%
% Technical questions _only_ to
% Gerald Murray (murray@hq.acm.org)
% ===============================================================
%
% For tracking purposes - this is V2.0 - May 2012

\PassOptionsToPackage{table}{xcolor}
\documentclass[sigconf,review,anonymous]{acmart}

%% Some recommended packages.
\usepackage{booktabs}   %% For formal tables:
                        %% http://ctan.org/pkg/booktabs
\usepackage{subcaption} %% For complex figures with subfigures/subcaptions
                        %% http://ctan.org/pkg/subcaption

\usepackage{mathptmx}
\usepackage{microtype}

%\usepackage{nccmath}
\usepackage{amsmath}
\usepackage{newtxmath}

\DeclareSymbolFont{largesymbolsCM}{OMX}{cmex}{m}{n}
\let\txsum\sum
\let\sum\relax
\DeclareMathSymbol{\sum}{\mathop}{largesymbolsCM}{"50}

\DeclareSymbolFont{tienlargesymbolsCM}{OMX}{cmex}{m}{n}
\let\txprod\prod
\let\prod\relax
\DeclareMathSymbol{\prod}{\mathop}{tienlargesymbolsCM}{"51}


\usepackage{balance}

\usepackage{ulem}

\usepackage[table]{xcolor}

\usepackage{listings}
\normalem
%\usepackage{latex8}
%\usepackage{times}
\usepackage{epsf}
\usepackage{ctable}
%\usepackage{latexsym}
%\usepackage{tweaklist}
%\usepackage{rotating}
%\usepackage{listings}
%\usepackage{alltt}
%\usepackage{fvrb-ex}
\usepackage{graphicx}
\usepackage{url}
\usepackage{float}
\usepackage{multicol}
%\floatstyle{boxed}
\restylefloat{figure}
\usepackage{hyperref}
\usepackage{comment}
\usepackage{paralist}

\usepackage{xspace}
\newcommand{\cf}{\hbox{\emph{cf.}}\xspace}
\newcommand{\deletia}{\ldots [deletia] \ldots}
\newcommand{\etal}{\hbox{\emph{et al.}}\xspace}
\newcommand{\eg}{\hbox{\emph{e.g.,}}\xspace}
\newcommand{\ie}{\hbox{\emph{i.e.,}}\xspace}
\newcommand{\st}{\hbox{\emph{s.t.}}\xspace}
\newcommand{\wrt}{\hbox{\emph{w.r.t.}}\xspace}
\newcommand{\viz}{\hbox{\emph{viz.}}\xspace}

\newcommand{\todo}[1]{\textcolor{red}{TODO: #1}\PackageWarning{TODO:}{#1!}}

%\usepackage{amsmath}
%\usepackage{booktabs}

\usepackage{multirow}
\usepackage{textcomp}

\usepackage[T1]{fontenc}
\usepackage{array}

\usepackage{fancyhdr}
\usepackage[yyyymmdd,hhmmss]{datetime}

\newtheorem{Definition}{Definition}
\newtheorem{Claim}{Claim}
\newtheorem{Lemma}{Lemma}
\newtheorem{Theorem}{Theorem}
\newtheorem{Property}{Property}

\newcommand{\code}[1]{{\small\textsf{#1}}}

%\newcommand{\hoan}[1]{{\color{green!70!black}\textbf{Hoan:}~#1}\xspace}
%\newcommand{\danny}[1]{{\color{blue!70!white}\textbf{Danny:}~#1}\xspace}
%\newcommand{\tien}[1]{{\color{violet!70!white}\textbf{Tien:}~#1}\xspace}
%\newcommand{\michael}[1]{{\color{cyan!70!white}\textbf{Michael:}~#1}\xspace}



\newcommand{\NumChanges}{\textcolor{black}{322K}\xspace}
\newcommand{\NumFiles}{\textcolor{black}{1M}\xspace}
\newcommand{\NumChangeFiles}{\textcolor{black}{291K}\xspace}
\newcommand{\NumChangesIntelliJ}{\textcolor{black}{XXX}\xspace}
\newcommand{\NumFilesIntelliJ}{\textcolor{black}{XXX}\xspace}
\newcommand{\NumDevelopers}{\textcolor{black}{108}\xspace}

\newcommand{\NumDevelopersSecondCorpus}{\textcolor{black}{13K}\xspace}
\newcommand{\NumberSLOCs}{\textcolor{black}{164M}\xspace}
\newcommand{\NumberChangeSLOCs}{\textcolor{black}{81M}\xspace}
\newcommand{\NumbersofChangeGraphNodes}{\textcolor{black}{3M}\xspace}


\newcommand{\NumRepos}{\textcolor{black}{88}\xspace}
\newcommand{\NumPatterns}{\textcolor{black}{17K}\xspace}
\newcommand{\NumRequests}{\textcolor{black}{451}\xspace}
\newcommand{\NumResponses}{\textcolor{black}{108}\xspace}
\newcommand{\RepeatedChangePatterns}{\textcolor{black}{XXX}\xspace}

\newcommand{\NumAllDevelopers}{\textcolor{black}{170,000+}\xspace}
\newcommand{\NumAllProjects}{\textcolor{black}{5,000+}\xspace}


\lstset{
    language={Java}, emph={},
    mathescape=false, escapeinside={/*@}{@*/},
    basicstyle=\small\sffamily,
    numberstyle=\small\sffamily,
    emphstyle=\bfseries,
    numbers=left, stepnumber=1, numbersep=-6pt,
    frame=single, xleftmargin=4pt, xrightmargin=4pt, framexleftmargin=0pt, framexrightmargin=0pt,  %xleftmargin=11pt
    columns=flexible, breaklines=true, showspaces=false, showstringspaces=true, showtabs=false, tabsize=2
}

\definecolor{deletedline}{RGB}{255,224,224}
\definecolor{addedline}{RGB}{224,255,224}
\definecolor{modifiedline}{RGB}{231,231,152}

%\lstset{
%    language={}, emph={},
%    mathescape=false, escapechar=@,
%    basicstyle=\scriptsize\sffamily,
%    numberstyle=\scriptsize\sffamily,
%    emphstyle=\bfseries,
%    numbers=left, stepnumber=1, %numbersep=7pt,
%    frame=single, xleftmargin=4pt, xrightmargin=4pt, framexleftmargin=0pt, framexrightmargin=0pt,  %xleftmargin=11pt
%    columns=flexible, breaklines=true, showspaces=false, showstringspaces=true, showtabs=false, tabsize=4
%}

%\def\alignauthor{%
%\end{tabular}%
%  \begin{tabular}[t]{p{0.88\auwidth}}\centering}%

%\makeatletter
%\let\@copyrightspace\relax  % clear copyright block!!!
%\makeatother

\begin{document}

%\special{papersize=8.5in,11in}
\setlength{\pdfpageheight}{\paperheight}
\setlength{\pdfpagewidth}{\paperwidth}

% Copyright
%\setcopyright{acmcopyright}
%\setcopyright{acmlicensed}
%\setcopyright{rightsretained}
%\setcopyright{usgov}
%\setcopyright{usgovmixed}
%\setcopyright{cagov}
%\setcopyright{cagovmixed}


% DOI
%\acmDOI{10.475/123_4}

% ISBN
%\acmISBN{123-4567-24-567/08/06}

\acmConference[FSE 2018]{International Symposium on Foundations of Software Engineering}{November 2018}{Orlando, USA}

%\acmPrice{\$15.00}

%
% --- Author Metadata here ---
%\conferenceinfo{WOODSTOCK}{'97 El Paso, Texas USA}
%\CopyrightYear{2007} % Allows default copyright year (20XX) to be over-ridden - IF NEED BE.
%\crdata{0-12345-67-8/90/01}  % Allows default copyright data (0-89791-88-6/97/05) to be over-ridden - IF NEED BE.
% --- End of Author Metadata ---

%------- TITLE ----------------------------------
%\title{Capture Relevancy Between Texts and API Elements via Software
%  Documentation with Vector Representation}

%Tien
%\title[]{Capturing Relevance Between English Words and API Elements via Vector Representation
%and Applications}

\title[]{Does BLUE Score Work for Source Code?}
%\subtitle {Draft version {\today} \currenttime}

%({\today} \currenttime)

%\numberofauthors{4} %  in this sample file, there are a *total*

% of EIGHT authors. SIX appear on the 'first-page' (for formatting
% reasons) and the remaining two appear in the \additionalauthors section.
%

%------------------------------------------
%\author{Hoan Anh Nguyen}
%\affiliation{Iowa State University, USA}
%\email{hoan@iastate.edu}

%\author{Tien N. Nguyen}
%\affiliation{University of Texas at Dallas, USA}
%\email{tien.n.nguyen@utdallas.edu}

%\author{Danny Dig}
%\affiliation{Oregon State University, USA}
%\email{digd@eecs.oregonstate.edu}

%\author{Michael Hilton}
%\affiliation{Carnegie Mellon University, USA}
%\email{mhilton@cmu.edu}

%---------------------------------------

% There's nothing stopping you putting the seventh, eighth, etc.
% author on the opening page (as the 'third row') but we ask,
% for aesthetic reasons that you place these 'additional authors'
% in the \additional authors block, viz.


%\date{30 July 1999}
% Just remember to make sure that the TOTAL number of authors
% is the number that will appear on the first page PLUS the
% number that will appear in the \additionalauthors section.


\begin{abstract}
Machine translation (MT) is a fast growing sub-field of computational linguistic. Until now, the most popular automatic metrics to measure the quality of MT is Bleu score. Lately, MT along with its Bleu metric has been applied to many Software Engineering(SE) tasks. In this paper, we studied Bleu score to validate its suitability for software engineering tasks. We showed that Bleu score does not reflect translation quality due to its weak relation with semantic meaning of the translated source codes. Specifically, an increase in Bleu score does not guarantee an improved in translation quality, and a good translation may have fluctuated Bleu score.  
\end{abstract}


%type references, method calls, and field references.

%\renewcommand{\shortauthors}{}

\setcopyright{none}

\settopmatter{printacmref=false, printfolios=false}

%\if@ACM@manuscript{false}

%\ACM@manuscriptfalse

%\printtopmatter{none}

\maketitle



%An important class of software engineering problems is centered around relation between the English text descriptions and
%relevant source code including API elements. A text is relevant to
%source code when it can be used to describe the functionality,
%behaviors, purposes, and related aspects of the code. Existing
%approaches capture the relevance relation between an English word and
%a code element by automatically tokenizing the name of the code/API
%element into English words, and a source file is considered as a
%collection of those words and the words from its comments, and then
%compared with textual software documents. This solution is not always
%ideal since in a program, a developer could have API elements' names
%with different textual values than the words in the text
%descriptions. In this paper, inspired by the Wor2Vec model for NLP,
%we develop a specialized neural network model, {\tool} that
%characterizes the APIs via a low-dimensional continuous vector space
%that encodes information on many contexts in which the APIs
%appear. With vector representations, {\tool} captures the relevance
%relations between English words and API elements.
%%and those amongwords and APIs.
%Our empirical evaluation showed that our vector
%representations reflect human knowledge of the
%relevance between words and API elements. We also demonstrated the usefulness of
%our model in two SE applications: searching API example code and mining single API
%mappings across software libraries.

%API documentation is a very crucial resource for developers in
%understanding various aspects on the API usages in software
%libraries. An interesting nature of such documentation is the presence
%of code elements embedded in natural language texts that explain their
%purposes, usages and mutual connections with others. Some recent work
%has explored that nature for different purposes such as discovering
%code elements in documents and generating summaries for classes and
%methods with context. However, none of the existing approaches is
%capable of further capturing semantic relations of code elements
%(i.e. embedded APIs hereafter) with related words and with other
%relevant APIs at the same time. In this paper, by considering an
%embedded API element as a word, we characterize this API via a
%low-dimensional continuous vector that encodes information on many
%contexts in which that API appears as possible. Our empirical study
%shows that the vector representation learned from a large corpus of
%documentation is capable of capturing semantic relations between API
%elements and words. That is, APIs with similar functionalities have
%similar embeddings; and APIs and related words are close to each other
%in the vector space without explicit word matching. Our experiment
%also suggests that the proposed representation for embedded API
%elements has promising potential for API code search.


\section{Introduction}
\label{sec:intro}

Statistical Machine Translation (SMT)~\cite{smtbook} is a natural
language processing (NLP) approach~that uses statistical learning to
derive the translation ``rules'' from a training data (called a {\em
  corpus}) and applies the trained model to translate a sequence from
the source language ($L_S$) to the target one ($L_T$). SMT produces
translated texts based on the statistical models whose parameters are
trained from a corpus of the corresponding texts in two languages. SMT
has been very successful in translating natural-language texts.
Google Translate~\cite{googletranslate} is a SMT-based tool~that can
accept inputs in 15 natural languages and allows the translation of a
word or a phrase into one of 53 languages.  Microsoft
Translator~\cite{mstranslator} also supports instant translation for
more than 40 languages.

The statistical machine translation community relies on the BLUE
metric (bilingual evaluation understudy) for the purpose of evaluating
SMT models and tools.
%
%what
%
BLUE metric or BLUE sore measures translation quality by the accuracy
of trxanslating $n$-grams to $n$-grams with various values of $n$
(phrases to phrases). BLEU was shown to be highly correlated with
human judgments in natural language MT systems~\cite{Papineni2002}.
Despite the criticism on BLUE metric, it has been remaining one of the
most popular automated and inexpensive metrics to evaluate the quality
of translation models.





Machine Translation (MT) is the use of computer program to translate
text or speech from one language to another. Bleu score evaluates the
quality of MT by calculating the modified n-grams precision and also
taking into account the length difference penalty. Traditionally, MT
is only applied to natural language, but now it is also used for
technical and programming language. One notable use of MT for SE tasks
is Code Migration. Even with that adaptation, SE community still
relies on Blue to evaluate the quality of MT. It is well known that
there is a significant difference between natural language and
programing language: programing language has structure, and
well-defined syntax. This leads to a question as whether Blue score is
suitable for SE task (Code Migration) or not. If it is, we could
continue to use it. Otherwise, we need another metric that is more
suitable for programing language. Hence, the answer to the question
above will help researchers and developers build and evaluate MT-based
Code Migration system better. Some has attempted to answer the
question by stating informal arguments toward the use of Bleu for SE
task \cite{}. However, up to date, there has not been any empirical
evidences to formally address the problem.

Bleu measures the lexical difference between machine generated code and referenced one. On the other hand, to measure the semantic similarity between them is the ultimate goal when evaluating quality of Code Migration system. 
 

Bleu was proved to be correlated with human judgments in natural language MT systems \cite {Papineni02}. However, Callison at el argued that we should not over-rely on Bleu score as an improvement in Bleu score is not sufficient nor necessary to show an in improvement in translation quality \cite {Callison06}. To validate the use of Bleu on SE tasks, we set up an experiment to manually judge the result of multiple MT systems and compare its to the Bleu score. Our result showed that Bleu score has weak correlation to human judgments across 

%\vspace{0.03in} {\em 1.}  \textbf{\code{BLEU}}. This is a popular
%metrics in SMT that measures translation quality by the accuracy of
%translating $n$-grams to $n$-grams with various values of $n$ (phrases
%to phrases):

% \[\code{BLEU} = BP.{e^{\frac{1}{n}(\log {P_1} + ... + \log {P_n})}}~\cite{bleu}\]
%where $BP$ is the {\em brevity penalty value}, which equals to 1 if
%the total length (i.e. the number of words) of the resulting sentences
%is longer than that of the {\em reference sentences} (i.e. the correct
%sentences in the oracle). Otherwise, it equals to the ratio between
%those two lengths. $P_i$ is the metrics for the overlapping between
%the bag of $i$-grams (repeating items are allowed) appearing in the
%resulting sentences and that of $i$-grams appearing in the reference
%sentences. Specifically, if $S^{i}_{ref}$ and $S^{i}_{trans}$ are the
%bags of $i$-grams appearing in the reference code and in the
%translation code respectively, $P_i$ = |$S^{i}_{ref}$ $\cap$
%$S^{i}_{trans}$|/|$S^{i}_{trans}$|. The value of \code{BLEU} is
%between 0-1. The higher it is, the higher the translation quality.

%Since $P_i$ represents the accuracy in translating phrases
%with $i$ consecutive words, the higher the value of $i$ is used, the
%better \code{BLEU} measures translation quality. For example, assume
%that a translation \code{Tr} has a high $P_1$ value but a
%low~$P_2$. That is, \code{Tr} has high word-to-word accuracy but low
%accuracy in translating 2-grams to 2-grams (e.g. the word order might
%not be respected in the result). Thus, using both $P_1$ and $P_2$ will
%measure \code{Tr} better than using only $P_1$. If
%translation~sen\-tences are shorter, \code{BP} is smaller and
%\code{BLEU} is smaller. If they are too long and more incorrect words
%occur, $P_i$ values are smaller, thus, \code{BLEU} is smaller. $P_i$s
%are computed for $i$=1-4.
 
\section{Research Questions and Hypothesis}
\subsection{RQ1}
Does bleu score reflect semantic meaning of translated source code?
\subsection{RQ2} 
If the answer to RQ1 is 'no', is Bleu correlated to Lexical representation of code?
\subsection{RQ3} 
If the answer to RQ1 is 'no', is Bleu correlated to Syntaxtical representation of code?
\subsection{Our hypothesis}
Our hypothesis is that bleu score does not measure well the closeness in term of semantics between the reference and translated source code. 
\section{Methodology}
\subsection{Proof of Hypothesis}
\subsection{Data Collection}
\subsection{Settings and Metrics}

\section{Evaluation}
\emph{2.} \textbf{String Edit Distance (SED):} This metric measures
effort that a user must edit in term of the code tokens
that need to be deleted/added in order to transform the
resulting code into the correct one. It is computed as:  $SED = \frac{EditDistance\left(s_R, s_T\right)}{length\left(s_R\right)}$ where $EditDistance\left(s_R, s_T\right)$ is the editing distance between each pair of the reference method $s_R$ and the translated method $s_T$; and the denominator is the total length of the referenced method. The metrics is also normalized in 0-1 range.

\emph{3.} \textbf{Tree Edit Distance (TREED):} This metric measures the difference between the Abstract Syntax Trees (AST) of referenced method and translated method. Specifically, the tree edit distance between two trees is calculated by number of operations (add/delete/replace/move) to make them identical. \cite{algorithm}. 
It is computed as:  $TREED = \frac{TreeEditDistance\left(AST_R, AST_T\right)}{CountNodes \left(AST_R\right)}$ where $TreeEditDistance\left(AST_R, AST_T\right)$ is the editing distance between two trees of referenced method $AST_R$ and the translated method $AST_T$; and the denominator is the total nodes in the tree of the referenced method.  The metrics is also normalized in 0-1 range.

\section{Proposal}
\section{Related Works}

\section{Conclusions}
This paragraph will end the body of this sample document.
Remember that you might still have Acknowledgments or
Appendices; brief samples of these
follow.  There is still the Bibliography to deal with; and
we will make a disclaimer about that here: with the exception
of the reference to the \LaTeX\ book, the citations in
this paper are to articles which have nothing to
do with the present subject and are used as
examples only.
%\end{document}  % This is where a 'short' article might terminate






% (36.8\% relatively higher). 
%We also built an Eclipse plugin that automatically adds the import
%statements of the required libraries for a given code snippet and sets
%up the Maven links to the repositories for those libraries.

%\section*{Acknowledgments}
%We would like to thank 
%Sarah Nadi,
%Stas Negara,
%Mihai Codoban,
%Sruti Srinivasa, % Sruti did not give feedback as she was busy with her own CHI submissions
%Zhen Yu,
%Ameya Ketkar,
%and Samantha Khairunnesa 
%for providing valuable feedback on earlier drafts of this paper.

% TODO: add here the Grant numbers

\newpage

\balance
%\bibliographystyle{abbrv}
%\citestyle{acmauthoryear}

\bibliographystyle{ACM-Reference-Format}

\setcitestyle{numbers,sort&compress}

%\setcitestyle{numbers,sort&compress}
\bibliography{fse18}



% That's all folks!
\end{document}
