\section{Hypothesis and Methodology}

  %Research Questions}

BLEU has been widely used in evaluating the translation quality of SMT
models in NLP. It has been empirically~validated to be correlated with
human judgments for the translation quality in natural-language
texts~\cite{Papineni2002}.
%
%was proved to be correlated with human judgments in natural
%language MT systews~\cite{Papineni02}, but no one has confirmed its
%validity for programming languages.
%
However, using BLEU metric to evaluate
%the migrated code from
the SMT-based code migration models would raise two issues. First,
programming languages
%are often used with automated tools, thus,
have well-defined syntaxes and program dependencies. Source code has
strict syntactic structure while natural-language texts are less
strict in that aspect.
%to enable creativity, poetry and metaphors.
%
%there are two reasons to be concern about the use of BLEU on SMT-based
%Code Migration system. First, programming language is much difference
%from natural language: Programming language is written for
%machines. So it has well-defined semantic and is
%unambiguous. Programming allows some variations, but the meaning is
%rigorous.
%On the other hand, natural languages is ambiguous, and has more
%relaxed semantics. Moreover, the naturalness of programming language
%is to give instructions to computer. Hence, it has strict structure
%and syntactic while natural language is more loosy in that aspect to
%enable creativity, poetry and metarphors.
%
Second, there is a gap between the purpose of BLEU and the task of
evaluating migrated code. BLEU measures the lexical precision of
translating results. However, when evaluating translated source code,
it is more important to consider the semantics/functionality of
the generated code.
%
The closer the semantics/functionality between the translated~code and
the reference code, the better translation quality is.
%
%Statically speaking, functionality of sources code is represented
%semantically by program dependence graph (PDG) as data and control
%dependencies among program entities.
%As a result, BLEU would fail to capture the semantic similarity
%between resulting code and reference code.

Due to those two intuitive reasons, we have the hypothesis that {\em
  BLEU score does not measure well the quality of translated results
  that is estimated based on the semantic/functionality similarity
  between the migrated source code and the reference one}. We prove
this by contradiction. First, we~assume that {\em BLUE
  measure well the quality of translated code}. From that assumption
statement, we derive two {\em two necessary conditions}:

%To validate this hypothesis, we aim to answer these following research
%questions:

{\bf Cond\_1 [Semantic Similarity].} BLEU score reflects well the
semantic similarity between the translated source code and the
reference code in the ground truth.

{\bf Cond\_2 [Model Comparison].} BLEU is effective in comparing the
translation quality of SMT-based migration models.

%{\bf RQ1:} Does BLEU score reflect well the semantic similarity between
%the translated source code and the reference code in the ground truth?

%{\bf RQ2:} Is BLEU effective in comparing the translation quality of SMT-based code migration models?

%Furthermore, we aim to answer a relevant research question:

%{\bf RQ3:} What is the alternative metric to measure the semantic accuracy
%of migrated code if BLEU is not effective in evaluating the translated results?
