\subsubsection{\textbf{Graph Vector Edit Distance (GVED)}}

Graph-based approaches in representing program semantics have become
popular. Among those approaches, PDGs are one of the most popular.  To
capture the graph structures in PDGs, we choose Exas~\cite{fase09}.
Exas is an efficient structural characteristic feature extraction
approach that approximates and captures the structure within the
fragments of artifacts~\cite{fase09}.  In our study, we use Exas as a
mean of computing the difference between two PDGs. Given a pair of
method in C\# which are need to be compared, their respective
(PDGs) are built.
% with some additional nodes from GROUM~\cite{fse09}.
Exas vectors will be computed on those graphs. In Exas, the
characteristic features are extracted from the patterns of elements of
the graphs. The code fragments are characterized by their counting
vectors of those features. The difference between two vectors reflects
the difference of two graphs.

%Therefore, distance of these counting vectors is considered the way to measure the sematic between code fragments.

Figure~\ref{fig:PDGs} shows the PDGs of the code fragments 1 and
2. These graphs are analysed by Exas, which focuses on two kinds of
patterns of structural information of the graph, called $\left(p,q\right)$-node
and $n$-path as can be seen in Table~\ref{tab:feature1} and
Table~\ref{tab:feature2}.

An efficient way to express the property ``having the same or similar
features'' is to use vectors. The characteristic vector of a fragment
is the occurrence-count vector of its features. That is, each position
in the vector is indexed for a feature and the value at that position
is the number of occurrences of that feature in the
fragment. Table~\ref{tab:featureIndex} shows the indexes of the
features, which are global across all vectors. Based on the occurrence
counts, the vectors for code fragment 1 and~2 are
$V_1$(1,1,1,1,1,0,1,0...) and $V_2$(1,1,0,1,1,1,1,1...),
respectively. Two fragments having the same feature sets and
occurrence counts will have the same vectors. The
vector similarity can be measured by a chosen vector distance such as
1-norm distance.

We introduce the formula for normalizing the result of vector edit
distance which is used in our experiments. The normalized
value is described as
\[GVED \left( V_1, V_2 \right) = 1 - \sum_{i=1}^{n} \frac{ \mid XV1_i - XV2_i \mid}{XV1_i + XV2_i}\]
where $n$ denotes the number of vector scalar, $V_1$ denotes the
counting vector of the first graph, $V_2$ denotes the counting vector
of the second graph; $XV1_i$ denotes the value of the ($i-th$) scalar
of $V_1$; and $XV2_i$ denotes the value of the ($i-th$) scalar of
$V_2$.

In this example in Figure~\ref{code1code2}, the value of GVED is
$GVED\left(V_1, V_2\right) = 1 - \frac{8 }{20} = 0.6 $.

\begin{figure}
\begin{lstlisting}[language=JAVA]
	Code 1:
	void foo(int i) {
		int j;
		if (i < 2) {
			j = 1;
		} else {
			j = 2;
		}
	}

	Code 2:
	void foo(int i) {
		int j;
		if (i < 2)
			j = i;	
		j = 2;
	}
\end{lstlisting}
\caption{Example of two code fragments}
\label{code1code2}
\end{figure}

\begin{figure}[t]
	\caption{An example: two PDGs represent code fragments 1 and 2}
	\includegraphics[scale=0.4]{img/Diagram_PDG.png}
	\centering
	\label{fig:PDGs}
\end{figure}

% Table generated by Excel2LaTeX from sheet 'Sheet1'
\begin{table}[t]
  \centering
  \caption{Feature Table of Code Fragment 1 in Exas}
  \scalebox{0.65}{
    \begin{tabular}{|l|l|l|l|l|r|}
    \toprule
    \textbf{Pattern} & \multicolumn{5}{c|}{\textbf{Feature of Code1}} \\
    \midrule
    \textit{\textbf{1-path}} & inputInt & intSmall2 & intEqual1 & intEqual2 & \multicolumn{1}{l|}{declareInt} \\
    \midrule
    \textit{\textbf{2-path}} & \multicolumn{1}{p{6.415em}|}{inputInt-intSmall2} & \multicolumn{1}{p{5.915em}|}{intSmall2-intEqual1} & \multicolumn{1}{p{6em}|}{intSmall2-intEqual2} & \multicolumn{1}{p{5.75em}|}{declareInt-intEqual1} & \multicolumn{1}{p{5em}|}{declareInt-intEqual2} \\
    \midrule
    \textit{\textbf{3-path}} & \multicolumn{1}{p{6.415em}|}{intputInt-intSmall2-intEqual1} & \multicolumn{1}{p{5.915em}|}{intputInt-intSmall2-intEqual2} &       &       &  \\
    \midrule
    \textit{\textbf{(p,q)-node}} & inputInt-0-1 & intSmall2-1-2 & intEqual1-2-0 & intEqual2-1-2 &  \\
    \bottomrule
    \end{tabular}%
	}
  \label{tab:feature1}%
\end{table}%

% Table generated by Excel2LaTeX from sheet 'Sheet1'
\begin{table}[t]
  \centering
	\caption{Feature Table of Code Fragment 2 in Exas}
	\scalebox{0.65}{
	    \begin{tabular}{|l|l|l|l|l|r|}
    \toprule
    \textbf{Pattern} & \multicolumn{5}{c|}{\textbf{Feature of Code2}} \\
    \midrule
    \textit{\textbf{1-path}} & inputInt & intSmall2 & intEqualInt & intEqual2 & \multicolumn{1}{l|}{declareInt} \\
    \midrule
    \textit{\textbf{2-path}} & \multicolumn{1}{p{6.415em}|}{inputInt-intSmall2} & \multicolumn{1}{p{5.915em}|}{intSmall2-intEqualInt} & \multicolumn{1}{p{6em}|}{declareInt-intEqual1} & \multicolumn{1}{p{5.75em}|}{declareInt-intEqual2} &  \\
    \midrule
    \textit{\textbf{3-path}} & \multicolumn{1}{p{6.415em}|}{intputInt-intSmall2-intEqualInt} &       &       &       &  \\
    \midrule
    \textit{\textbf{(p,q)-node}} & inputInt-0-1 & intSmall2-1-1 & intEqualInt-2-0 & intEqual2-1-0 &  \\
    \bottomrule
    \end{tabular}%
	}
  \label{tab:feature2}%
\end{table}%

% Table generated by Excel2LaTeX from sheet 'Sheet1'
\begin{table}[htbp]
	\centering
	\caption{Feature Indexing}
	\scalebox{0.75}{
	\begin{tabular}{|cccc|}
		\toprule
		\multicolumn{1}{|l|}{\textbf{Feature}} & \multicolumn{1}{l|}{\textbf{Index}} & \multicolumn{1}{l|}{\textbf{Counted in Code1}} & \multicolumn{1}{l|}{\textbf{Counted in Code2}} \\
		\midrule
		\multicolumn{1}{|l|}{inputInt} & \multicolumn{1}{r|}{1} & \multicolumn{1}{r|}{1} & \multicolumn{1}{r|}{1} \\
		\midrule
		\multicolumn{1}{|l|}{intSmall2} & \multicolumn{1}{r|}{2} & \multicolumn{1}{r|}{1} & \multicolumn{1}{r|}{1} \\
		\midrule
		\multicolumn{1}{|l|}{intEqual1} & \multicolumn{1}{r|}{3} & \multicolumn{1}{r|}{1} & \multicolumn{1}{r|}{0} \\
		\midrule
		\multicolumn{1}{|l|}{intEqual2} & \multicolumn{1}{r|}{4} & \multicolumn{1}{r|}{1} & \multicolumn{1}{r|}{1} \\
		\midrule
		\multicolumn{1}{|l|}{declareInt} & \multicolumn{1}{r|}{5} & \multicolumn{1}{r|}{1} & \multicolumn{1}{r|}{1} \\
		\midrule
		\multicolumn{1}{|l|}{intEqualInt} & \multicolumn{1}{r|}{6} & \multicolumn{1}{r|}{0} & \multicolumn{1}{r|}{1} \\
		\midrule
		\multicolumn{1}{|p{5.75em}|}{inputInt-intSmall2} & \multicolumn{1}{r|}{7} & \multicolumn{1}{r|}{1} & \multicolumn{1}{r|}{1} \\
		\midrule
		\multicolumn{1}{|l|}{intEqual2-1-0} & \multicolumn{1}{r|}{8} & \multicolumn{1}{r|}{0} & \multicolumn{1}{r|}{1} \\
		\midrule
		\multicolumn{4}{|c|}{To be continued} \\
		\bottomrule
	\end{tabular}%
	}
	\label{tab:featureIndex}%
\end{table}%
