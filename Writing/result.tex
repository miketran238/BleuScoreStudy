\section{Empirical Results on BLEU Scores}
\label{sec:bleuresult}

%In this section, we present the empirical results to validate our
%hypothesis that \textit{BLEU is not effective in evaluating
%  translation quality of source code migration task}.

\subsection{Correlation between BLEU scores and semantic scores}
To justify the first part of our hypothesis, BLEU score does not reflect well
the semantic accuracy of results translated by a particular model,
we show the relation between BLEU scores and human judgments via semantic scores.
We use Pearson's correlation coefficient~\cite{geek_2015} to gauge
how strong their relation is. The correlation coefficient has a value
between [-1, 1], where 1 indicates the strongest positive relation, -1
indicates the strongest negative relation, and 0 indicates no relation.

%TODO BLEU, n-gram, n=???
Fig.~\ref{fig:BleuSemlpSMT} and Fig.~\ref{fig:BleuSemMppSMT} show the
scatter plots between two metrics: BLEU and Semantic. Each point
represents the scores of a pair of methods where its $x$-axis value is
for BLEU scores and $y$-axis value is for semantic scores. The
correlation coefficient between BLEU and semantic scores for the model
mppSMT is 0.523 and for the model lpSMT is 0.570. These positive values
are closer to 0.5 than to 1.0. This means there is a {\em positive but weak
relation} between BLEU score and semantic score. The weak correlations %help me to check grammar
between the metrics on the results translated by lpSMT and mppSMT are
demonstrated in Fig.~\ref{fig:BleuSemlpSMT} and Fig.~\ref{fig:BleuSemMppSMT}.

%\emph{Observation 1:}

%TODO mark what we want to talk about on figures

\begin{figure}
\caption{BLEU scores vs Semantic scores (lpSMT model)}
\centering
\includegraphics[width=2.9in]{img/bleuvssemantic_lpSMT.png}
\label{fig:BleuSemlpSMT}
\end{figure}

\begin{figure}
\caption{BLEU scores vs Semantic scores (mppSMT model)}
\centering
\includegraphics[width=2.9in]{img/bleuvssemantic_mppSMT.png}
\label{fig:BleuSemMppSMT}
\end{figure}

\subsubsection{{\bf Higher BLEU does not necessarily reflect higher~semantic accuracy}}

In Fig.~\ref{fig:BleuSemlpSMT}, for many specific values of BLEU, the
corresponding semantic scores can spread out for a wide range. For
instance, the BLEU score of 0.75, the corresponding semantic scores
are from 0.25 to 1.
% \textbf{Ngoc: can u mark in the figure please}.
Thus, from this observation, we conclude that the results migrated by
these models with {\em high BLEU scores might not achieve high semantic
scores}.
%There are two reasons for this.
%

%In our sample set, these results can fall in two main cases.
There are two main reasons for this result in our dataset.  First, the
translated methods might have multiple correct phrases, but in an
incorrect order, those methods can be incorrect, even not compilable.
%useless and justified as so in human judgment.
%
For example, in Fig.~\ref{fig:issueexample2}, the translated method
misplaces the position of '\{', making the method have a low
semantic score, however, it has high BLEU score.
%
%Another reason for this implication is that resulting method does not capture the important
%program elements.
In other cases, the migrated results are incomplete code missing the
elements that are trivial for the translation model, yet important
with respect to the syntactic rules of the target language. For
example, the result contains mostly keywords and separators such as
\code{if}, \code{public}, \code{()}, but misses out the important
program elements such as function calls or variable names. In this
case, it will have low semantic score while having a moderate to high
BLEU score. These circumstances indicate the weakness of BLEU metric
in evaluating the translated results in programming language where
syntax rules are well-defined.



%\emph{Observation 2:} For a fixed value of Semantic score, there can
%be many associated BLEU values. Specifically, in the model lpSMT, with
%a Semantic Score of 1, the BLEU scores can vary greatly between 0-1,
%which is reflected on the top horizontal line of dots in the
%Figure~\ref{fig:BleuSemlpSMT}. Similarly, in
%Figure~\ref{fig:BleuSemMppSMT}, with the Semantic Score of 1, the BLEU
%scores are in the range of 0.5 to 1.

\subsubsection{{\bf Higher semantic accuracy is not necessarily reflected by
higher BLEU score}} In the results translated by mppSMT
(Fig.~\ref{fig:BleuSemMppSMT}), for a particular value of the semantic
score, there can be many corresponding BLEU values spreading out in
a~wide range. Specifically, with the semantic score of 1, the BLEU
scores can vary from 0.5 to 1.0. Thus, it can be concluded
that for this model, the migrated code achieving {\em higher semantic score does not necessarily have higher BLEU~score}.

%From this observation, it can be implied that translated code can have
%low BLEU score, but high Semantic score. This can be explained by two
%reasons.
From our data, we observe that there are two main reasons leading to
that result.
%
First, a translated method can use code structure different from the
reference one to perform the same
functionality. Fig.~\ref{fig:mppSMT_example} shows an example that got
a maximum semantic score, but has low BLEU score of 0.4. In this
example, the translated method uses a \code{for} loop instead of a
\code{foreach} loop as in the reference code. The second reason is the
whitespace issue. For example, in Fig.~\ref{fig:mppSMT_example}, the
translated code has the tokens \code{IsSimilar(}, but the reference
code has \code{IsSimilar (}. The former is interpreted as one token
while the latter is interpreted as two. This situation reduces
the precision on phrases, but the human subject still evaluated the
result with a high semantic score. By this experiment, we also
empirically verify that the focus on the lexical precision of BLEU
makes it unable to capture other code-related aspects such as program
dependencies that contribute to program semantics. This might lead
to the ineffectiveness of BLEU in reflecting the semantic accuracy of
translated results.

In brief, {\em BLEU does not reflect well the
semantics of source code, and it is not suitable for evaluating the 
semantic accuracy of the result from an SMT-based code migration model}.

%TODO split translated results
\begin{figure}[t]
\centering
%\begin{lstlisting}[basicstyle=\small\sffamily, stepnumber=1, numbers=left, language=Java, aboveskip=1pt,  belowskip=1pt, numbersep=-5pt]
\lstinputlisting[basicstyle=\scriptsize\sffamily,language=Java]{mppExample.cs}
%\end{lstlisting}
\caption{Translated results and the corresponding reference code}
\label{fig:mppSMT_example}
\end{figure}


\subsection{The Use of BLEU in Comparing Models}
\subsubsection{Theoretical Models}
To prove the second part of our hypothesis, we conduct study to see if an improvement in BLEU score over models reflects the improvement in translation quality represented by human judgments of semantic score. For a same set of 375 original Java methods, the two models mppSMT and mppSMT2 generates two sets of 375 translated methods. Each set has its own BLEU scores and Semantic scores. 
From the 375 pairs, we have 73 pairs that have identical BLEU scores but are different in term of lexical. 26 out of 73 (\textbf{35\%}) have lower semantic score. On average, semantic score of model mppSMT2 is lower than mppSMT's by 0.16. Given that our semantic scores have only 5 pivots range from 0 to 1, a deduction of 0.16 is a large margin as it nearly reduces the semantic score by one pivot point.  
The model mppSMT2 would swap the positions of incorrect phrases in term of lexical. However, those phrases may still have semantic information as the model mppSMT is capable of produce code that are semantically correct but lexically incorrect. There are many cases that they are semantically incorrect but still help the whole translated code syntactically correct or provide users with good starting point for the migration task. We have showed that a theoretical model can achieve identical BLEU score but has lower translation quality compared to another. 
\subsubsection{Practical Models}
Using the same experiment set up, we have two sets of 375 translated methods from two models GNMT and mppSMT. From the sample, we choose pairs of results such that their GNMT's BLEU scores are higher than mppSMT's BLEU scores. There are 215 of such pairs. Among them, 79 pairs have lower semantic score. Then, we perform t-test with alpha confident level of 0.95 on that subset to see if their GNMT's semantic scores are also significantly higher or not. Our null hypothesis is: GNMT's semantic score are higher than mppSMT's one with confident level of 0.95. The results show a t-value of \textbf{-9.1} which is lower than the critical point of -1.97. It meant we would reject the null hypothesis that GNMT's semantic scores are higher than mppSMT's ones. Our result shows that an improvement in BLEU score does not lead to an improvement in translation quality. 
%To validate the use of BLEU in comparing different SMT-based code migration systems, we conduct study to see if an improvement in BLEU score over cross models reflects the improvement in translation quality represented by human judgments of semantic score between those models. For a same set of 375 original Java methods, the two models GNMT and mppSMT generates two sets of 375 translated methods. Each set has its own BLEU scores and Semantic scores.
%We prove that an improvement in BLEU score does not sufficient nor necessary lead to an improvement in semantic score by showing:
%
%1. From the sample, we choose pairs of results such that their GNMT's BLEU scores are higher than mppSMT's BLEU scores. Then, we perform t-test with alpha confident level of 0.95 on that subset to see if their GNMT's semantic scores are also higher or not. The results show a t-value of ... It meant we would reject the null hypothesis that GNMT's semantic scores are higher than mppSMT's ones. So, it can be concluded that an improvement in BLEU score is not sufficient to achieve a higher semantic score. 
%
%2.  From the sample, we choose pairs of results such that their mppSMT's semantic scores are higher than GNMT's semantic scores. Then, we perform t-test with alpha confident level of 0.95 on that subset to see if their mppSMT's BLEU scores are also higher or not. The results show a t-value of ... It meant we would reject the null hypothesis that mppSMT's BLEU scores are higher than GNMT's ones. So, it can be concluded that an improvement in BLEU score is not necessary to achieve a higher semantic score. 

%We ignored pairs of translated results if they have the same BLEU score or Semantic score (136 of the cases). For the remaining results, we found out that in \textbf{34\%} of the cases, the change in BLEU score contradicts the change in Semantic score. It means an improvement in BLEU score leads to a decrease in Semantic score and vice versa. In other words, if one function is translated by two migration models, one-third of the time, the result which has higher BLEU score actually has lower translation quality than the other.

Because of the results above, BLEU is not reliable to use in comparing different SMT-based migration models.




