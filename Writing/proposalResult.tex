\subsection{Proposal result}
To evaluate RUBY, we conduct the same experiment with other four metrics that shows the correlation of RUBY with semantic scores on two model mppSMT and lpSMT. Based on human evaluating presented on previous sections, the semantic values of these two model spread in range of [0..1]. Therefore, RUBY's efficiency will be tested on a variety of testcase number.  	


Figure \ref{fig:RubySemlpSMT} and figure \ref{fig:RubySemMppSMT} stimulate the scatter plots between RUBY and semantic scores on two models mppSMT and lpSMT respectively. Generally, RUBY has high correlation with semantic scores. As our experient result, the correlation coefficient between RUBY and semantic score for model mppSMT is \textbf{0.862} and for model lpMSMP is \textbf{0.836}. In statistics, these values indicate a strong uphill linear relationship between two quantitive metrics. That means one metric could be predicted by the other with high confidence. Therefore, RUBY reflects well to the semantic score.


As can be seen from the scatter plot for lpSMT (figure \ref{fig:RubySemMppSMT}), there is a moderately strong, positive, linear association between the two variables with a few outliers.  
Beside, the scatterplot for mppSMT (figure \ref{fig:RubySemMppSMT}) shows a strong, positive, linear association between RUBY and Semantic score. There don't appear to be any outliers in the data.
%Specifically, a correlation of +0.8 implies that when 	

Although RUBY has a strong correllation coefficient, it is still lower than VGED scored performed in section 5 due to the dataset scope of each method. While VGED is only computed with must-graph-parsed code, RUBY doesn't depend on the code, even we can not parse PDGs and ASTs. This implites that there exist a RUBY score for any given code. On the other hand, in pairwise comparison with TREED, SED and BLEU, RUBY performs much better.
	    
  			