\subsection{Empirical Results for {\model}}
\subsubsection{Correlation of {\model} and Semantic Scores}
To evaluate the effectiveness of {\model} in comparison to BLEU, we
conducted experiments in the same manner with BLEU on three models
GNMT, mppSMT, and lpSMT.
%{\model}'s efficiency was tested on a large number
%of methods translated by those models. 

%\begin{figure}[t]
%\caption{{\model} vs semantic score (lpSMT)}
%\centering
%\includegraphics{img/rubyvssem_lpSMT.png}
%\label{fig:RubySemlpSMT}
%\end{figure}
%
%\begin{figure}[t]
%\caption{{\model} vs semantic score (mppSMT)}
%\centering
%\includegraphics{img/rubyvssem_mppSMT.png}
%\label{fig:RubySemMppSMT}
%\end{figure}

Our result shows that the correlation coefficients between {\model}
and semantic score are 0.764, 0.815, and 0.747 for the three models
GNMT, mppSMT, and lpSMT respectively. They are higher comparing to the
correlation of BLEU of 0.658, 0.523, and 0.570.
%, {\model}'s results are much higher.
%In statistics,
These values indicate a strong positive linear relationship between
two quantitative metrics. Thus, they show the effectiveness of
{\model} in reflecting semantic accuracy.
%of translated results.

%Figures~\ref{fig:RubySemlpSMT} and~\ref{fig:RubySemMppSMT} show the
%scatter plots between {\model} and semantic scores on two models mppSMT
%and lpSMT, respectively. For lpSMT (figure~\ref{fig:RubySemlpSMT}), 
%there is a moderately strong, positive, linear association between the 
%two variables with a few outliers. Beside, figure~\ref{fig:RubySemMppSMT} 
%shows a strong, positive, linear association between {\model} and semantic 
%score for the results translated by mppSMT. There are no outliers in the data. 
%This indicates the result is consistent.

%In general, {\model} has high correlation with semantic scores. As our
%experiment result, the correlation coefficient between {\model} and
%semantic score for model mppSMT is \textbf{0.862} and for the model
%lpMSMP is \textbf{0.836}. In statistics, these values indicate a
%strong uphill linear relationship between two quantitative
%metrics. That means one metric could be predicted by the other with
%high confidence. For example, an increase of 0.5, {\model} score can
%be interpreted as an increase of 0.4 in term of Semantic score. Based
%on that information, developers can tune the system in an incremental
%manner.
%Specifically, a correlation of +0.8 implies that when 	

In general, the correlation coefficient between {\model} and semantic score cannot be 
stronger than GRS or even TRS. However, {\model} scores are available for any pair 
of the translated result and the reference code, while GRS and TRS are applicable only to 
a subset of translated results. 
%
%In our sample data, the sizes of these applicable set of methods are quite small.

%, for lpSMT they are \textbf{75/375} and \textbf{123/375}, and these figures
%for mppSMT are \textbf{239} and \textbf{292} respectively. 
%The summary results about correlation with semantic score is presented in figure \ref{fig:summary}

%While GRS is
%only computed with the migrated code with sufficient semantic
%information, {\model} does not depend on the code, even we can not parse
%PDGs and ASTs. This implies that there exist a {\model} score for any
%given code. On the other hand, in comparison with TRS, STS
%and BLEU, {\model} always outperforms the other three metrics.
\subsubsection{The Use of RUBY in Comparing Models}
We conducted the same experiment on {\model} as we did with BLEU on cross models. 
\textbf{mppSMT vs p-mppSMT}\\
On these two models that have identical BLEU scores, {\model} is able to identify their difference in translation quality estimated by semantic similarity. Specifically, the model mppSMT's average {\model} score is higher than p-mppSMT by \textbf{0.15} which nearly equals to the difference in term of semantic score (0.17). We also perform t-test on the data set to see if there is a significant difference in {\model} scores. With confident level of 0.95, we get the p-value of reject the null hypothesis that mppSMT's {\model} score is indifferent with p-mppSMT's ones.  
%Comparing models GNMT and mppSMT, we found out that in \textbf{97.7\%} of the cases, the change in {\model} score indicates the change in the same direction for Semantic score. It means an improvement in {\model} score equivalents an improvement in Semantic scores and vice versa. More interestingly, 83 of 88 cases (\textbf{94.3\%}) if a pair of translated results have the same Semantic score, they also have identical {\model} score. 
As the results showing, {\model} is a reliable metric to use in comparing different SMT-based migration models. 	

%\begin{figure}[t]
%	\caption{The correlation coefficients between metrics and semantic score}
%	\includegraphics[scale=0.85]{img/summary.jpg}
%	\centering
%	\label{fig:summary}
%\end{figure}
