\subsection{Proposal result}
To evaluate RUBY, we conduct the same experiment with other four metrics to show the correlation of RUBY with semantic scores on two model mppSMT and lpSMT. RUBY's efficiency will be tested on large number of translated methods generated by those two models. 

Figure \ref{fig:RubySemlpSMT} and figure \ref{fig:RubySemMppSMT} stimulate the scatter plots between RUBY and Semantic scores on two models mppSMT and lpSMT respectively. As can be seen from the scatter plot for lpSMT (figure \ref{fig:RubySemMppSMT}), there is a moderately strong, positive, linear association between the two variables with a few outliers.  
Beside, the scatter plot for mppSMT (figure \ref{fig:RubySemMppSMT}) shows a strong, positive, linear association between RUBY and Semantic score. There are no outliers in the data. It means the result is consistent. 

In general, RUBY has high correlation with Semantic scores. As our experiment result, the correlation coefficient between RUBY and semantic score for model mppSMT is \textbf{0.862} and for model lpMSMP is \textbf{0.836}. In statistics, these values indicate a strong uphill linear relationship between two quantitative metrics. That means one metric could be predicted by the other with high confidence. For example, an increase of 0.5 {\model} score can be interpreted as an increase of 0.4 in term of Semantic score. Based on that information, developer can tune the system in an incremental manner and fast turn-over. 


%Specifically, a correlation of +0.8 implies that when 	

Although RUBY has a strong correlation coefficient, it is still lower than VGED scored performed in section 5 due to the data set scope of each method. While VGED is only computed with must-graph-parsed code, RUBY doesn't depend on the code, even we can not parse PDGs and ASTs. This implies that there exist a RUBY score for any given code. On the other hand, in pairwise comparison with TREED, SED and BLEU, RUBY always outperforms the other three metrics.
	    
  			