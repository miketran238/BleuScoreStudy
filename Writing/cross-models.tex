\subsection{The Use of BLEU in Comparing Models}
To validate the use of BLEU in comparing different SMT-based code migration systems, we conduct study to see if an improvement in BLEU score over cross models reflects the improvement in translation quality represented by human judgments of semantic score between those models. For a same set of 375 original Java methods, the two models GNMT and mppSMT generates two sets of 375 translated methods. Each set has its own BLEU scores and Semantic scores.
We prove that an improvement in BLEU score does not sufficient nor necessary lead to an improvement in semantic score by showing:

1. From the sample, we choose pairs of results such that their GNMT's BLEU scores are higher than mppSMT's BLEU scores. Then, we perform t-test with alpha confident level of 0.95 on that subset to see if their GNMT's semantic scores are also higher or not. The results show a t-value of ... It meant we would reject the null hypothesis that GNMT's semantic scores are higher than mppSMT's ones. So, it can be concluded that an improvement in BLEU score is not sufficient to achieve a higher semantic score. 

2.  From the sample, we choose pairs of results such that their mppSMT's semantic scores are higher than GNMT's semantic scores. Then, we perform t-test with alpha confident level of 0.95 on that subset to see if their mppSMT's BLEU scores are also higher or not. The results show a t-value of ... It meant we would reject the null hypothesis that mppSMT's BLEU scores are higher than GNMT's ones. So, it can be concluded that an improvement in BLEU score is not necessary to achieve a higher semantic score. 
