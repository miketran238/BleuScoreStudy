\subsection{The Use of BLEU in Comparing SMT Migration Models}

%TODO complete and describe t-test results
To confirm the difference in the semantic scores for the two
models even though their results have the same BLEU~score, we performed the
paired $t$-test\cite{geek_2015} on 375 results translated by mppSMT and
p-mppSMT with the confidence level of
0.95 and a null hypothesis that {\em the semantic scores of two models
are equal}. With such setting for this data set, the $t$-critical value is 1.967,
and the confidence interval for the difference between the mean semantic scores
of two models is $\pm0.04$. It means the difference must be larger than 0.04 to
be considered significant. The $p$-value (the probability
that the results occurred by chance) of our test is 2.81E-85, which is much
smaller than significant level of 0.05. Thus, we rejected the null hypothesis and concluded
that the semantic scores of two models are different with the confidence
level of~95\%.

%TODO gonna be removed
Additionally, in our experiment on the results translated by mppSMT
and p-mppSMT, the average BLEU score for these models are
identical. Meanwhile, the overall semantic score for p-mppSMT is 0.62,
which is lower than the average for mppSMT of 0.88. This means that
the deviant of BLEU in comparing the two models is nearly two
levels of semantic score lower, despite that they have identical BLEU
score.

%To evaluate the abilities of BLEU in comparing the
%translation quality of models, we show that the two models mppSMT and p-mppSMT have different semantic score despite having identical BLEU score. We perform the paired t-test on the sample with confident level of 0.95. Our null hypothesis is the semantic scores of two models are equal. The p-value of the t-test is 2.81E-85
%which is much smaller than 0.05. Thus we reject our null hypothesis and conclude that the semantic scores of two models are different with confident level of 95\%.


%Tien
%In our dataset, there are two main reasons for these results. First,
%in some results of mppSMT, these phrases are lexically different those
%in the reference one, but they may carry correct syntactical or
%semantic information, which makes these results achieve high semantic
%score. However, for the results translated by p-mppSMT and having
%similar BLEU scores, these meaningful phrases are swapped, thus, are
%placed in incorrect locations. This leads to significantly decreasing
%in semantic scores. For example, in Fig.~\ref{fig:mppSMT_example}, the
%result translated by p-mppSMT which has the same BLEU score with the
%result of mppSMT. Meanwhile, the result of mppSMT is semantically
%correct, however, the result from p-mppSMT is not compilable.
%
%Furthermore, although several result of mppSMT might contain phrases
%that are lexically incorrect, these phrases have the information that
%provides developers clues to fix the translated one. However, these
%phrases in the corresponding results translated by p-mppSMT are moved
%into new positions, which also makes the semantic score of these
%results are much lower than that of mppSMT. For example,...
%\textbf{NGOC, example for this reason}

Examining the translated code, we found that p-mppSMT produces the
code with lower semantic accuracy due to the swapping of the locations
of the expressions and/or statements in the code. That leads to the
completely different program semantics and even not compilable code. For
example, in Fig.~\ref{fig:mppSMT_example}, the phrase with the method
name \code{isSimilar} is placed at the end of line 27, making the code
incorrect. Therefore, the semantic score of such result is lower
than the result from mppSMT.

%




%\subsubsection{Theoretical Models}
%To prove the second part of our hypothesis, we conduct study to see if an improvement in BLEU score over models reflects the improvement in translation quality represented by human judgments of semantic score. For a same set of 375 original Java methods, the two models mppSMT and mppSMT2 generates two sets of 375 translated methods. Each set has its own BLEU scores and Semantic scores.
%From the 375 pairs, we have 73 pairs that have identical BLEU scores but are different in term of lexical. 26 out of 73 (\textbf{35\%}) have lower semantic score. On average, semantic score of model mppSMT2 is lower than mppSMT's by 0.16. Given that our semantic scores have only 5 pivots range from 0 to 1, a deduction of 0.16 is a large margin as it nearly reduces the semantic score by one pivot point.
%The model mppSMT2 would swap the positions of incorrect phrases in term of lexical. However, those phrases may still have semantic information as the model mppSMT is capable of produce code that are semantically correct but lexically incorrect. There are many cases that they are semantically incorrect but still help the whole translated code syntactically correct or provide users with good starting point for the migration task. We have showed that a theoretical model can achieve identical BLEU score but has lower translation quality compared to another.
%\subsubsection{Practical Models}

%Using the same experiment set up, we have two sets of 375 translated methods from two models GNMT and mppSMT. From the sample, we choose pairs of results such that their GNMT's BLEU scores are higher than mppSMT's BLEU scores. There are 215 of such pairs. Among them, 79 pairs have lower semantic score. Then, we perform t-test with alpha confident level of 0.95 on that subset to see if their GNMT's semantic scores are also significantly higher or not. Our null hypothesis is: GNMT's semantic score are higher than mppSMT's one with confident level of 0.95. The results show a t-value of \textbf{-9.1} which is lower than the critical point of -1.97. It meant we would reject the null hypothesis that GNMT's semantic scores are higher than mppSMT's ones. Our result shows that an improvement in BLEU score does not lead to an improvement in translation quality.

%To validate the use of BLEU in comparing different SMT-based code migration systems, we conduct study to see if an improvement in BLEU score over cross models reflects the improvement in translation quality represented by human judgments of semantic score between those models. For a same set of 375 original Java methods, the two models GNMT and mppSMT generates two sets of 375 translated methods. Each set has its own BLEU scores and Semantic scores.
%We prove that an improvement in BLEU score does not sufficient nor necessary lead to an improvement in semantic score by showing:
%
%1. From the sample, we choose pairs of results such that their GNMT's BLEU scores are higher than mppSMT's BLEU scores. Then, we perform t-test with alpha confident level of 0.95 on that subset to see if their GNMT's semantic scores are also higher or not. The results show a t-value of ... It meant we would reject the null hypothesis that GNMT's semantic scores are higher than mppSMT's ones. So, it can be concluded that an improvement in BLEU score is not sufficient to achieve a higher semantic score.
%
%2.  From the sample, we choose pairs of results such that their mppSMT's semantic scores are higher than GNMT's semantic scores. Then, we perform t-test with alpha confident level of 0.95 on that subset to see if their mppSMT's BLEU scores are also higher or not. The results show a t-value of ... It meant we would reject the null hypothesis that mppSMT's BLEU scores are higher than GNMT's ones. So, it can be concluded that an improvement in BLEU score is not necessary to achieve a higher semantic score.

%We ignored pairs of translated results if they have the same BLEU score or Semantic score (136 of the cases). For the remaining results, we found out that in \textbf{34\%} of the cases, the change in BLEU score contradicts the change in Semantic score. It means an improvement in BLEU score leads to a decrease in Semantic score and vice versa. In other words, if one function is translated by two migration models, one-third of the time, the result which has higher BLEU score actually has lower translation quality than the other.

%Because of the results above, BLEU is not reliable to use in comparing different SMT-based migration models.

In our work, we additionally conducted another experiment in the same
manner on two pairs of SMT-based migration models: lpSMT~\cite{fse13}
and GNMT~\cite{gnmt}, and mppSMT~\cite{ase15} and
GNMT~\cite{gnmt}. The results from our $t$-tests allowed us to
conclude that an improvement in BLEU score is not sufficient nor
necessary to reflect a higher semantic score with the confidence of
0.95. Thus, the experimental results also invalidate the conditions. All
results can be found on our website~\cite{ruby-website}.

In brief, {\em BLEU is not effective in comparing SMT-based code
  migration models}. Finally, because both of the above necessary
conditions are false, we can conclude that {\em BLEU score does not
  measure well the quality of translated results}.

%In brief, we conclude that \textit{BLEU is not effectiv in comparing
%  the translation quality of SMT-based code migration models}.
